\thispagestyle{plain}

\section*{Kurzfassung}
Zur Entwicklung neuartiger Bauteile mit größerer Effizienz und Leistungsfähigkeit stellt die Grenzfläche zwischen antiferromagnetischen Übergangsmetalloxiden und organischen Komponenten einen Ansatzpunkt da.
Hierzu sind die Kenntnisse über die grundlegenden physikalischen Eigenschaften des Substrates und der oragnischen-anorganischen Grenzfläche entscheidend.
In der Arbeit wird eine erste Charakterisierung verschiedener antiferromagnetischer Filme vorgenommen.
Hierbei wird ein polarer Nickeloxidfilm der (111)-Orientierung verwendet, ebenso wie ein Film aus Wüstit mit einer (100)-Oberfläche.
Mit Hilfe der Beugung niederenergetischer Elektronen sowie Photoelektronenspektroskopie werden die Substrate untersucht.
Anschließend wird eine Monolage aus Pentacen aufgebracht und mittels Photoemissionsorbitaltomographie näher betrachtet.
Erkennbar ist, dass sich die Moleküle auf der Nickeloxidoberfläche nicht selbst anordnen, wohingegen beim Wüstit dies zutrifft und das niedrigste zuvor unbesetzte Molekülorbital besetzt wird.

\section*{Abstract}
\begin{foreignlanguage}{english}
For the design of new devices, that are more efficient and have a higher performance, the interface of organic materials and antiferromagnetic transition metal oxides is a starting point.
Therefor the knowledge about the physical and chemical properties of the organic-inorganic interface is important.
This work will contain a first characterisation of antiferromagnetic substrates.
It will deal with a polar (111) nickeloxide and a (100) wüstite surface.
Low energy electron diffraction and photoelectron spectroscopy will be the tools of choice to do so.
On the characterised substrate a single layer of pentacene molecules will be applied and investigated with the photoemission orbital tomography.
It can be seen that the molecules do not self-arrange on the polar surface, but they do on the wüstite and the lowest unoccupied molecular orbital will be occupied.
\end{foreignlanguage}
