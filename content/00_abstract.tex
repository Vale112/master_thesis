\thispagestyle{plain}

\section*{Kurzfassung}
Zur Entwicklung neuartiger Bauteile mit größerer Effizienz und Leistungsfähigkeit stellt die Grenzfläche zwischen antiferromagnetischen Übergangsmetalloxiden und organischen Komponenten einen Ansatzpunkt da.
Hierzu sind die Kenntnisse über die grundlegenden physikalischen Eigenschaften des Substartes und der oragnischen-anorganischen Grenzfläche entscheidend.
In der Arbeit wird eine erste Charakterisierung verschiedener antiferromagnetischer Filme vorgenommen.
Hierbei wird ein polarer Nickeloxidfilm der (111)-Orientierung verwendet, ebenso wie ein Film aus Wüstit mit einer (100)-Oberfläche.
Mit Hilfe der Beugung niederenergetischer Elektronen, sowie Photoelektronenspektroskopie werden die Substrate untersucht.
Anschließend wird eine Monolage aus Pentacen aufgebracht und mittels Photoemissionsorbitaltomographie näher betrachtet.
Erkennbar ist, dass sich die Moleküle auf der Nickeloxidoberfläche nicht selbstanordnen, wohingegen beim Wüstit dies zutrifft und das LUMO besetzt wird.

\section*{Abstract}
\begin{foreignlanguage}{english}

\end{foreignlanguage}
