\thispagestyle{plain}

\section*{Kurzfassung}
Bei der Suche nach neuen Materialkombinationen für elektronische Bauteile mit höherer Effizienz und Leistungsfähigkeit sind zuletzt organische Komponenten sowie antiferromagnetische Übergangsmetalloxide in den Fokus der Forschung geraten.
Für zukünftige Anwendungen sind die Kenntnisse über die grundlegenden physikalischen Eigenschaften des Substrates und der organisch-anorganischen Grenzfläche entscheidend.
In dieser Arbeit wird eine erste Charakterisierung verschiedener antiferromagnetischer Filme sowie deren Interaktion mit Pentacen vorgenommen.
Hierbei werden ein polarer Nickeloxidfilm der (111)-Orientierung und ein Film aus Wüstit mit einer (100)-Oberfläche verwendet und mit Hilfe der Beugung niederenergetischer Elektronen sowie Photoelektronenspektroskopie untersucht.
Anschließend wird eine Monolage aus Pentacen aufgebracht und mittels Photoemissionsorbitaltomographie näher betrachtet.
Während auf der \ce{NiO}-Oberfläche keine Ordnung der Moleküle erkennbar war, konnte bei der Wüstit-Oberfläche mehrere Orbitale inklusive Ladungstransfer in ein vorher unbesetztes Orbital beobachtet werden.

\section*{Abstract}
\begin{foreignlanguage}{english}
    Doing the search for new materials for electronic devices with a higher efficiency and performance, organic compounds as well as antiferromagnetic transition metal oxides comes into the focus of research.
    For applications in the future it is important to have the knowledge about the physical and chemical properties of the organic-inorganic interface.
    This work will contain a first characterisation of different antiferromagnetic substrates with and without pentacene on top.
    It will deal with a polar (111) nickeloxide and a (100) wüstite thin films which are characterised by low energy electron diffraction and photoelectron spectroscopy.
    On the characterised substrate a single layer of pentacene molecules will be applied and investigated with the photoemission orbital tomography.
    It can be seen that the molecules do not self-arrange on the polar surface, but they do on the wüstite where some molecular orbitals get identified as well as due to charge transfer the lowest unoccupied molecular orbital will be occupied.
\end{foreignlanguage}
