\chapter{Einleitung}
    Die Anzahl an Menschen die digitale Geräte wie etwa ein Smartphone nutzen steigt stetig~\cite{Statista}.
    Damit steigt auch der Energiebedarf der Digitalisierung an.
    Steigerung gibt es aktuell kaum noch in der Effizenz, Bauteilgröße und Minimierung der Leistungsaufnahme.

    Das untersuchte Sytem ist interressant für die IT.
    So könnten Moleküle z.B. optisch angeregt werden und damit Spinwellen im AFM auslösen. 
    An andere Stelle werden diese Spinwellen von einem weiteren Molekül aufgenommen und in ein einfach messbares Signal umgewandelt~\cite{SINFONIA}.
    Da dieser Prozess keinen Transport von Elektronen wie in herkömlichen auf Halbleiter basierenden technischen Geräten entfällt die Joulsche Wärme.

    Dieses System kann also hinsichtlich dem großen Energiebedarf zur Kühlung in Rechenzentern diesen senken.
    Auch ist die Methode schneller, da die Anregung optisch geschieht und der Transport ohne Elektronenbewegung stattfindet.
    Aktuelle Entwicklungen der Halbleiterindustrie auf CMOS Basis sind komm noch in ihrer Größe minimierbar, auch hier könnte das System zu einer kleineren Bauweise beitragen. 
    Somit wären gleich mehrer Probleme der IT-Industrie gleichzeitig gelöst.

    NiO zeigt bereits Eigenschaften des Katalisators~\cite{kunz_chemisorption_1985}, eine Erweiterung und Verbesserung mit Hilfe der Moleküle wäre denkbar.
\begin{itemize}
    \item Herkömmliche HL ablösen -> neue materialien
    \item schneller
    \item kleiner
    \item effizienter
    \item Aufgrund der Größe sind Oberflächeneffekte stark präsent
    \item Erforschung der elektronischen und magnetischen Eigenschaften (und deren Kombination)
    \item the ferromagnetic/organic interface of organic     based spintronic devices has been target of research because evidence was found for its excessive impact on the overall device performances [15, 16, 17]. (DJ)
    \item By combining momentum-resolved photoemission measurements with ab initio calculations that are performed in the framework of density functional theory, MOT enables the unambiguous identification of distinct molecular orbitals and provides information about the azimuthal orientation of the molecules [28, 29]. Precise knowledge about the electronic structure at the interface is crucial for an understanding of the molecule-substrate interactions. (DJ)
    \item Übergangsmetalloxide
    \item organisch basierte Bauteile
    \item Ladungstransport ist langsam
    \item Chemisorption ermöglicht es die Eigenschaften der Grenzfläche zu tunen
    \item geringerer Energieaufwand für die Anregung
    \item Warum MOT? - Da dann im engen VB die Zustände Orbitalen zugeornet werden können.
\end{itemize}