\chapter{Einleitung}
    Die Anzahl an Menschen die digitale Geräte, wie etwa ein Smartphone nutzen steigt stetig~\cite{Statista}.
    Damit steigt auch der Energiebedarf der Digitalisierung an.
    Steigerung gibt es aktuell kaum noch in der Effizenz, Bauteilgröße und Minimierung der Leistungsaufnahme.
    Im Gegensatz dazu stehen neue Materialkombinationen aus organischen und anorganischen Materialien.
    So könnten Moleküle z.B. optisch angeregt werden und damit Spinwellen im Antiferromagneten auslösen. 
    An andere Stelle werden diese Spinwellen von einem weiteren Molekül aufgenommen und in ein einfach messbares Signal umgewandelt~\cite{SINFONIA}.
    Die Frequenzskala auf der dies passiert liegt im \si{\tera\hertz}-Bereich~\cite{AFM_5}.
    Damit wird die Geschwindigkeit der heutigen Zeit nochmal deutlich erhöht.
    Da dieser Prozess, der Spinwellen keinen Transport von Elektronen wie in herkömlichen auf Halbleiter geschieht entfällt die Joulsche Wärme.
    Folglich muss eine geringere Energie aufgewendet werden und somit die Effizenz steigert.
    Auch die Größe der Moleküle führt zu einer kleineren Bauweise.
    Die Verwendung von leichten Atomen in diesen reduziert des Weiteren auch das Gewicht der entsprechenden Bauteile, was besonders für mobile Geräte besonders interessant ist.
    Auch der immer größere Bedarf und Engpässen an Halbleitern \cite{Idealo} und damit steigenden Preisen sorgen für die Nachfrage nach Alternativen.
    Bisher standen vermehrt ferromagnetisch-organische Grenzflächen bei der Forschung nach Spin-elektronischen Bauteilen im Fokus~\cite{ma-DJ}.
    Um die Vorteile der antiferromagnetischen Materialien und eine gute Anpassbarkeit zu erhalten sind Übergangsmetalloxide ideale Kandidaten.
    Ihre elektronischen Eigenschaften, insbesondere ihre Austrittsarbeit lässt sich einfach modifizieren.
    Was besonders die Energiedifferenz zwischen Fermi und HOMO beziehungsweise LUMO betrifft, welche ein entscheidenen Schlüssel für die Anwendung darstellt~\cite{5A_4}.
    Heutzutage werden Übergangsmetalloxide bereits eingesetzt um als Zwischenschichten zwischen Elektrode und organischer Schicht in oragnischen Bauteilen den Kontaktwiderstand zu reduzieren und die Energieniveauanpassung der Moleküle zu ermöglichen~\cite{IF_11}.
    Pentacene wird bereits heute in einigen organisch elektronischen Bauteilen eingesetzt \cite{5A_4}.
    Das Aufbringen der Moleküle auf ein antiferromagnetisches Substrat, wie dem Eisenmonooxid kann zu neuen Phänomenen und Eigenschaften führen.
    Die Erweiterung auf die Kombination mit einer polaren und somit reaktiven antiferromagnetischen Oberfläche wie die \ce{NiO} (111)-Oberfläche stellt eine neue Idee dar~\cite{cappus_hydroxyl_1993}.
    % Um die Art der Interaktion zu bestimmen um die Performance zu verbessern eignet sich die Molekülorbitaltomographie besonders gut.


    % Dieses System kann also hinsichtlich dem großen Energiebedarf zur Kühlung in Rechenzentern diesen senken.
    % Auch ist die Methode schneller, da die Anregung optisch geschieht und der Transport ohne Elektronenbewegung stattfindet.
    % Aktuelle Entwicklungen der Halbleiterindustrie auf CMOS Basis sind komm noch in ihrer Größe minimierbar, auch hier könnte das System zu einer kleineren Bauweise beitragen. 
    % Somit wären gleich mehrer Probleme der IT-Industrie gleichzeitig gelöst.
    % 
    % NiO zeigt bereits Eigenschaften des Katalisators~\cite{kunz_chemisorption_1985}, eine Erweiterung und Verbesserung mit Hilfe der Moleküle wäre denkbar.
\begin{itemize}
    \item Aufgrund der Größe sind Oberflächeneffekte stark präsent
    \item Erforschung der elektronischen und magnetischen Eigenschaften (und deren Kombination)
    \item the ferromagnetic/organic interface of organic based spintronic devices has been target of research because evidence was found for its excessive impact on the overall device performances [15, 16, 17]. (DJ)
    \item By combining momentum-resolved photoemission measurements with ab initio calculations that are performed in the framework of density functional theory, MOT enables the unambiguous identification of distinct molecular orbitals and provides information about the azimuthal orientation of the molecules [28, 29]. Precise knowledge about the electronic structure at the interface is crucial for an understanding of the molecule-substrate interactions. (DJ)
    \item Ladungstransport ist langsam
    \item Chemisorption ermöglicht es die Eigenschaften der Grenzfläche zu tunen
    \item geringerer Energieaufwand für die Anregung
    \item Warum MOT? - Da dann im engen VB die Zustände Orbitalen zugeornet werden können.
    \item Warum NiO - weil AFM -> Spinwellen, und Isolator keine Elektronenbewegung keine Joulsche Wärme
    \item In recent years there has been growing interest in the field of thin organic semiconducting films due to their successful application in optical and electronic devices, such as organic light emitting diodes (OLED) organic field effect transistors (OFET). \cite{Uni-Tübingen}
    \item PEN in OFETS (\url{https://www.sciencedirect.com/science/article/abs/pii/S0065272518300357})
    \item Organische sind umweltfreundlich \cite{scholl_chapter_2018}
    \item Große Anpassbarkeit der Moleküle \cite{scholl_chapter_2018}
    \item "The interface is the device" von H. Kroemer (Nobelpreis Physik 2000)
    \item optische Stimulation, Antwort auf äußere Felder, ihre chemischen und physikalischen eigenschaften zu verknüpfen mit den eigenschaften des Interfaces \cite{IF_16}
    \item externen input umwandeln um es nutzbar zu magnetischen
    \item Kontrolle der magnetischen Eigenschaften des Interfaces/dünnen Filme durch die Wechselwirkung mit dem Molekülen (Hybridisierung) \cite{IF_16}
    \item AFm größere stabilität und schnellere Dynamiken als FM \cite{AFM_1}
    \item alles kleiner FM schwierig dann genau auszurichten und so, da sorgt AFM für größere Stabilität 
    \item AFM können elektronischen Spin-Strom in magnonischen Spin Strom umwandeln (exp. gezeigt bei insolierenden AFM) \cite{AFM_1}
    \item Magnonen brauchen keine Stromversorgung (schwer bei Chips zu ralisierene) und haben da kein Teilchentransport kein verlust durch Joulsche-Wärme (Isolatoren) \cite{AFM_3}
    \item Magnonenfrequenz im THz Bereich realisierbar und damit größer als in Fm \cite{AFM_5}
    \item erste theoretische Berechnungen zeigen lokale spin-polarisation durch Moleküle auf AFM \cite{AFM_2} -> Spinterface
    \item Vergleinerung von Leseköpfen, welche GMR nutzen durch verwendung von AFM Molekülen und Ferromagneten \cite{bagrets_single_2012} für selbstanordung uninteressant
\end{itemize}

Eine neue Forschungsidee des europäischen Forschungsprojekt Sinfonia beschäftigt sich mit der Kopplung zwischen Molekülen und Antiferromagneten.
        Ferner sollen diese genutzt werden um Spinwellen, so genannte Magnonen zu generieren und zu detektieren.
        Dies könnte auf Grund der Magnonenfrequenz im \si{\tera\hertz}-Bereich neue Geschwindigkeitsrekorde in der Datenverabeitung bringen~\cite{SINFONIA}.
        In Folge dessen werden in dieser Arbeit erste Charakterisierungen für potenzielle Substrate und Moleküle untersucht.