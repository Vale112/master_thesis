\chapter{Einleitung}
    % Wieso
    Die Anzahl an Menschen, die digitale Geräte wie etwa ein Smartphone nutzen, steigt stetig~\cite{Statista}.
    Damit steigt gleichermaßen der Energie- und Materialbedarf in der Technologiebranche.
    Allerdings gibt es kaum noch Ausbaupotential von Effizienz, Bauteilgröße und Minimierung der Leistungsaufnahme in der herkömmlichen Halbleitertechnik.
    Schon voreinigen Jahren wurden neue Materialklassen, eine Kombination aus organischen und anorganischen Materialien, in modernen Smartphones verbaut~\cite{MAC}.
    Diese OLEDs (\textit{organic light emitting diodes}) gehören wie die OFETs (\textit{organic field effect transistors}) zu der organischen Halbleitertechnik.
    Durch die Integration dieser Komponenten in bestehende Systeme und die Überwindung von Grenzen der heutigen Halbleitertechnik, hat diese Forschungsbereich in den letzten Jahrzehnten immer mehr Aufmerksamkeit auf sich gezogen~\cite{Uni-Tübingen}.
    Ebenfalls sorgt der immer größere Bedarf und damit steigende Preise und Engpässe bei Halbleitern für die Nachfrage nach alternativen Materialien~\cite{Idealo}.

    % Grenzfläche
    Bei solchen organisch-anorganischen Grenzflächen sind die Eigenschaften der beiden Materialen eng miteinander verknüpft.
    Die Zustände der Moleküle hybridisieren mit den Zuständen der Oberfläche, sodass die Eigenschaften dieser Grenzfläche für das Funktionsverständnis an Bedeutung gewinnen.
    Dabei ist die Verwendung organisch-anorganischen Materialkombinationen viel versprechend, aufgrund der großen Vielzahl an zur Verfügung stehenden Materialien und der damit verbundenen Anpassbarkeit an die Anforderungen der Industrie.
    Es hat sich gezeigt, dass sich die magnetischen Eigenschaften der Molekül, wie auch der Grenzfläche durch die entsprechende Wahl der Substanzen und Schichtdicken beeinflussen lassen~\cite{IF_16}.
    Die Erforschung der elektronischen und magnetischen Eigenschaften und deren Kombination dieser Schichtsysteme stehen im Fokus vieler Forschungsprojekte.
    Die große Anpassbarkeit von organischen Molekülen in Bezug auf ihre Struktur und Zusammensetzung geht einher mit ihren elektronischen Eigenschaften~\cite{scholl_chapter_2018}.
    Auch die geringe Größe der Moleküle ermöglicht eine weiter Minimierung der elektronischen Komponenten.
    Auf die Übergangsmetalloxide trifft ebenfalls das Argument der großen Anpassbarkeit hinsichtlich ihrer geometrischen und elektronischen Struktur zu~\cite{5A_4}.
    So lassen sich also vom Substrat wie auch den Molekülen die Eigenschaften entsprechend anpassen. % damit sie die gewünschten elektronischen Eigenschaften zeigen.

    Ziel ist es eine externe Eingabe über die Moleküle umzuwandeln und an das Substrat weiter zu geben, um diese dann nutzbar zu machen.
    Hierzu soll mittels optischer Stimulation oder einem äußeren Felde die chemische und physikalische Eigenschaft des Moleküls beeinflusst werden.
    Aufgrund der Verknüpfung mit den Eigenschaften der Grenzfläche können diese hier weiter an das Substrat gegeben werden~\cite{IF_16}.
    Durch die gezielte Manipulation von hybridisierten Zuständen der organisch-anorganischen antiferromagnetischen Systeme lassen sich zum Beispiel Spinwellen mithilfe optischer Anregung erzeugen und auslesen~\cite{SINFONIA}.

    % Spin und AFM -> TMOs
    Die Kopplung an die magnetischen Eigenschaften bietet den Vorteil, dass keine Joulsche-Wärme produziert wird, da bei einem Transport der Informationen über Spins keinen Ladungstransport erforderlich ist~\cite{AFM_3}.
    Der Transport der Informationen über Spinwellen ist dadurch gleichzeitig schneller.
    Dieser Ansatz wird in der aktuellen Forschung verfolgt um die geringe Spin-Bahn-Kopplung innerhalb der Moleküle mit den Spins im Festkörper zu verbinden.
    Dabei standen bisher vermehrt ferromagnetisch-organische Grenzflächen bei der Forschung nach Spin-elektronischen Bauteilen im Fokus~\cite{ma-DJ, AFM_5}.
    Nachdem für chemisorbierte Moleküle auf Ferromagneten bereits Spin polarisierte Zustände gefunden wurden \cite{IF_16} ergeben erste theoretische Berechnungen dies für antiferromagnetische Oberflächen.
    Dort zeigen sich bei der Kombination von Molekülen und Antiferromagneten lokale Spinpolarisationen an der Grenzfläche, welche die Kerneigenschaft von Spinterfaces darstellt~\cite{AFM_2}.
    Hinzukommend können isolierende Antiferromagneten einen elektronischen Spin-Strom in einen magnonischen Spin-Strom umwandeln~\cite{AFM_1}.
    Da die Frequenzen von Spinwellen in Antiferromagneten im \si{\tera\hertz}-Bereich liegen, kann dadurch die Taktung elektronischer Geräte im Vergleich zur heutigen Zeit noch einmal deutlich erhöht werden~\cite{AFM_5}.
    Dabei zeigt die Verwendung von antiferromagnetischen Übergangsmetalloxiden eine größere Stabilität hinsichtlich ihrer Magnetisierung und schnellere Dynamiken als es für Ferromagneten bekannt ist~\cite{AFM_1}.
    
    Der Riesenmagnetowiderstand (GMR, \textit{giant magnetoresistance}) dient in den Lesköpfen von Festplatten zur Auslesung der magnetischen Domänen.
    Bisher wurden Antiferromagneten meist nur als Fixierungslage, die die Magnetisierungsrichtung der hart magnetisierbaren ferromagnetischen Lage gegen ein äußeres Feld festhält, eingesetzt.
    Neuerdings können Antiferromagneten diese Lage direkt darstellen, was zu einer Reduzierung der Bauteilgröße führt.
    Ferner wird die bisherige nicht magnetische anorganische Schicht durch eine organische Lage ersetzt~\cite{bagrets_single_2012}.

    Für Anwendungen ist die Lage der Molekülzustände hinsichtlich der Fermikante des Substrates von großer Bedeutung, da dies den Ladungstransfer beeinflusst.
    Dabei stellt diese Energieniveauanpassung bei der Erforschung von organisch-anorganischen Systemen einen entscheidenden Schlüssel dar.
    Eine besondere Rolle spielt dabei die Austrittsarbeit des verwendeten Substrates sowie die Ionisationsenergie des Moleküls~\cite{5A_3}.
    Durch gezielte Eingriffe in den Präparationsprozess lässt sich zum Beispiel bei Übergangsmetalloxiden die Austrittsarbeit anpassen~\cite{5A_4}.
    Heutzutage werden Übergangsmetalloxide bereits eingesetzt um als Zwischenschichten zwischen Elektrode und organischer Schicht in organischen Bauteilen den Kontaktwiderstand zu reduzieren und die Energieniveauanpassung der Moleküle zu ermöglichen~\cite{IF_11}.
    Um die Vorteile antiferromagnetischer Materialien zu nutzen und dabei die Anpassbarkeit des Kontaktwiderstand nicht zu verlieren, sind Übergangsmetalloxide eine Möglichkeit.

    Zur Identifizierung vielversprechende Kandidaten aus Molekül-Substart-Kombinationen ist zunächst die Wechselwirkung zwischen Molekülen und Substraten zu verstehen sowie deren elektronische und geometrische Struktur an der Grenzfläche.
    So sind vor allem die Zustände nahe der Fermikante wie auch die Molekülzustände für die spätere Anwendung entscheiden.
    Regelmäßig anordnende Moleküle auf Oberflächen stellen dabei einen Teilbereich dar, der z.B. für die Datenspeicherung von Nutzen sein könnte, da die gezielte Manipulation einzelner Moleküle ermöglicht wird.
    Um solche selbstanordnenden Moleküle hinsichtlich ihrer elektronischen Struktur zu untersuchen ist die Verwendung der Photoemissionorbitaltomographie eine passende Methode.
    Dabei kann sie Aufschluss über die Bandstruktur des Substartes wie auch die eindeutige Identifizierung der beteiligten Molekülorbitale und deren Ausrichtung auf dem Substrat ermöglichen~\cite{MM_2, MM_5}.
    Ebenfalls gibt sie dabei Aufschluss über die Energieniveauanpassung, welche für den Ladungsaustausch entscheidend ist.
    Bevor sich Spin tragende Eigenschaften genauer angeschaut werden können, müssen zunächst vielversprechende Kandidaten aus Molekül-Substart-Kombinationen gefunden werden.

    Nickeloxid und Eisenmonooxid sind antiferromagnetische Übergangsmetalloxide.
    In dieser Arbeit wurde Nickeloxid auf eine Goldoberfläche aufgebracht, da ihr Gitterkonstanten um nur \SI{5}{\percent} voneinander abweichen.
    Dabei bilden sich eine instabile, polare und reaktive (111)-Oberfläche aus, wobei die Spins in der obersten Lage eine gemeinsame Orientierung einnehmen~\cite{cappus_hydroxyl_1993}.
    Das Eisenmonooxid wurde auf eine passivierte Eisenoberfläche aufgebracht, wodurch sich ein nur \SI{6}{\percent}-ige Abweichung der Gitterkonstanten ergibt.
    Diese (100)-Oberfläche ist stabil und die Spins sind an der Oberfläche abwechselnd orientiert.
    Das untersuchte Pentacen wird aufgrund seiner hohen Elektronenbeweglichkeit heute schon in einigen organisch elektronischen Bauteilen eingesetzt~\cite{5A_4, 5A_13}.
    Dieses aus fünf aneinander verschmolzenen Phenylringen bestehende Molekül \cite{MM_2}, zeigte bereits auf zahlreichen Oberflächen die Selbstanordnung~\cite{5A_4, 5A_1, 5A_6, 5A_10, 5A_5, 5A_9}.
    
    % So werden für den Einsatz in organischen Feldeffekttransistoren eine Schicht aus isolierendem Material, wie zum Beispiel \ce{NiO} oder \ce{FeO} und Molekülen, wie dem Pentacen benötigt~\cite{5A_13}.

    % NiO zeigt bereits Eigenschaften des Katalisators~\cite{kunz_chemisorption_1985}, eine Erweiterung und Verbesserung mit Hilfe der Moleküle wäre denkbar.
% \begin{itemize}
%     \item the ferromagnetic/organic interface of organic based spintronic devices has been target of research because evidence was found for its excessive impact on the overall device performances [15, 16, 17]. (DJ)
%     \item PEN in OFETS (\url{https://www.sciencedirect.com/science/article/abs/pii/S0065272518300357})
% \end{itemize}

% Wie schon der Nobelpreisträger H. Kroemer im Jahre 2000 sagte, ist die Grenzfläche das Bauteil (\textit{"The interface is the device"}).
% Die Verwendung von Molekülen, die aus leichten Atomen bestehen, reduziert des Weiteren das Gewicht der entsprechenden Bauteile, was besonders für mobile Geräte interessant ist.
% Durch den Verzicht auf giftige Schwermetalle und andere Substanzen sind solche organischen Materialien umweltfreundlicher als ihre Konkurrenten der heutigen Technik~\cite{scholl_chapter_2018}.
% Durch die Verwendung von Magnonen entfällt die Stromversorgung.
% Hierdurch wird zum einen Energie gespart zum anderen fällt ein Problem bei der Chipproduktion, der Spannungsversorgungsleitungen innerhalb des Chips, weg~\cite{AFM_3}.
% Heutzutage finden Antiferromagneten als eigene interaktive Lage bei kleineren Leseköpfen von Festplatten Anwendung, welche auf dem Riesenmagnetowiderstand (GMR, \textit{giant magnetoresistance}) beruhen.
% Sie dienen nicht mehr nur als Fixierungslage für eine ferromagnetische Schicht sondern eigenständig in Kombination mit Molekülen und einer ferromagnetischen Lage~\cite{bagrets_single_2012}.
% Der Energieaufwand für Anregungen in den organischen Materialien ist geringer als in den anorganischen Materialien und kann somit einfacher realisiert werden.