\chapter{Einleitung}
    Das untersuchte Sytem ist interressant für die IT.
    So könnten Moleküle z.B. optisch angeregt werden und damit Spinwellen im AFM auslösen. 
    An andere Stelle werden diese Spinwellen von einem weiteren Molekül aufgenommen und in ein einfach messbares Signal umgewandelt.
    Da dieser Prozess keinen Transport von Elektronen wie in herkömlichen auf Halbleiter basierenden technischen Geräten entfällt die Joulsche Wärme.

    Dieses System kann also hinsichtlich dem großen Energiebedarf zur Kühlung in Rechencentern senken.
    Auch ist die Methode schneller, da die Anregung optisch geschieht.
    Eine kleinere Bauweise wäre auch Denkbar. 
    Somit wären gleich mehrer Probleme der IT-Industrie gleichzeitig gelöst.

    NiO zeigt Eigenschaften des Kathalisators \cite{kunz_chemisorption_1985}.