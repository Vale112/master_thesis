\chapter{Einleitung}
    % Wieso
    Die Anzahl an Menschen, die digitale Geräte wie etwa ein Smartphone nutzen, steigt stetig~\cite{Statista}.
    Damit steigt auch der Energie- und Materialbedarf der Digitalisierung an.
    Steigerung gibt es aktuell kaum noch in der Effizienz, Bauteilgröße und Minimierung der Leistungsaufnahme.
    Wie schon der Nobelpreisträger H. Kroemer im Jahre 2000 sagte, ist die Grenzfläche das Bauteil (\textit{"The interface is the device"}).
    Im Gegensatz zur herkömmlich Halbleiterbasierten Technik stehen neue Materialkombinationen aus organischen und anorganischen Materialien.
    Durch ihre Anwendung in der organischen Halbleitertechnik wie den OLEDs (\textit{organic light emitting diodes}) und OFETs (\textit{organic field effect transistors}) ergibt sich ein großes Interesse auf diesem Gebiet in den letzten Jahren~\cite{Uni-Tübingen}.
    So könnten Moleküle z.B. optisch angeregt werden und damit Spinwellen im Antiferromagneten auslösen. 
    An anderer Stelle werden diese Spinwellen von einem weiteren Molekül aufgenommen und in ein einfach messbares Signal umgewandelt~\cite{SINFONIA}.
    Da die Frequenzen von Spinwellen in Antiferromagneten im \si{\tera\hertz}-Bereich liegen, kann dadurch die Taktung elektronischer Geräte im Vergleich zur heutigen Zeit noch einmal deutlich erhöht werden~\cite{AFM_5}.
    Bei diesem Prozess entfällt die Joulsche Wärme, da Spinwellen kein Transport von Elektronen, wie es bei herkömmlichen auf Halbleitern basierenden Komponenten stattfindet~\cite{AFM_3}.
    Folglich muss für dieselben Prozesse eine geringere Energie aufgewendet werden.
   
    % Grenzfläche
    Die Verwendung von Grenzflächen von organisch-anorganischen Materialien ist viel Versprechend aufgrund der großen Vielzahl an zur Verfügung stehenden Materialien.
    So sind die Eigenschaften von organischen Materialien auf Oberflächen mit dessen Eigenschaften eng verknüpft.
    Zusätzlich sorgt die räumliche Einschränkung zu weiteren Zuständen und Oberflächeneffekte sind stark ausgeprägt.
    Die Erforschung der elektronischen und magnetischen Eigenschaften und deren Kombination dieser Schichtsysteme stehen im Fokus vieler Forschungsprojekte.
    Ziel ist es eine externe Eingabe umzuwandeln und an das Substrat weiter zu geben um dies dann nutzbar zu machen.
    Dies über magnetische Eigenschaften zu realisieren, bietet den Vorteil, dass keine Ladung transportiert werden muss und somit die Geschwindigkeit erhöht werden kann.
    Hierzu soll mittels optischer Stimulation oder einem äußeren Felde die chemische und physikalische Eigenschaft des Moleküls beeinflusst werden.
    Aufgrund der Verknüpfung mit den Eigenschaften der Grenzfläche können diese hier weiter transportiert oder umgewandelt werden~\cite{IF_16}.
    Der Energieaufwand für Anregungen in den organischen Materialien ist geringer als in den anorganischen Materialien und kann somit einfacher realisiert werden.
    Es hat sich gezeigt, dass sich die magnetischen Eigenschaften der Molekül, wie auch der Grenzfläche durch die entsprechende Wahl der Substanzen und Schichtdicken, die mit der Hybridisierung einher gehen, beeinflussen lassen~\cite{IF_16}.
    Dabei hat sich besonders der Prozess der Chemisorption als tragender Beitrag erwiesen.

    % Moleküle
    Die große Anpassbarkeit von organischen Molekülen in Bezug auf ihre Struktur und Zusammensetzung sind viel Versprechend.
    Einher gehen damit ihre elektronischen Eigenschaften, die somit für den jeweiligen Prozess optimiert werden können~\cite{scholl_chapter_2018}.
    Auch die geringe Größe der Moleküle ermöglicht eine weiter Minimierung der elektronischen Komponenten.
    Die Verwendung von Molekülen, die aus leichten Atomen bestehen, reduziert des Weiteren auch das Gewicht der entsprechenden Bauteile, was besonders für mobile Geräte interessant ist.
    Durch den Verzicht auf giftige Schwermetalle und andere Substanzen sind solche organischen Materialien umweltfreundlicher als ihre Konkurrenten der heutigen Technik~\cite{scholl_chapter_2018}.
    Auch der immer größere Bedarf und damit steigende Preise und Engpässe bei Halbleitern sorgen für die Nachfrage nach alternativen Materialien~\cite{Idealo}.

    % AFM+TMOs
    Bisher standen vermehrt ferromagnetisch-organische Grenzflächen bei der Forschung nach Spin-elektronischen Bauteilen im Fokus~\cite{ma-DJ}.
    Um die Vorteile antiferromagnetischer Materialien, der Anpassbarkeit vom Kontaktwiderstand und Struktur zu erhalten, sind Übergangsmetalloxide ein ideale Möglichkeit.
    Insbesondere ihre Austrittsarbeit lässt sich einfach modifizieren.
    Dies betrifft die Energiedifferenz zwischen Fermi und dem höchsten besetzten beziehungsweise niedrigsten unbesetzten Molekülorbital, welche ein entscheidender Schlüssel für die Anwendung darstellt~\cite{5A_4}.
    Heutzutage werden Übergangsmetalloxide bereits eingesetzt um als Zwischenschichten zwischen Elektrode und organischer Schicht in organischen Bauteilen den Kontaktwiderstand zu reduzieren und die Energieniveauanpassung der Moleküle zu ermöglichen~\cite{IF_11}.
    Die Verwendung von antiferromagnetischen Übergangsmetalloxiden zeigt eine größere Stabilität hinsichtlich ihrer Magnetisierung und schnellere Dynamiken als es für Ferromagneten bekannt ist~\cite{AFM_1}.
    Durch die Reduktion in ihrer Größe lassen sich die magnetischen Momente der Ferromagneten schwieriger Ausrichten als bei Antiferromagneten.
    Hinzukommend können isolierende Antiferromagneten einen elektronischen Spin-Strom in einen magnonischen Spin-Strom umwandeln~\cite{AFM_1}.
    Durch die Verwendung von Magnonen entfällt die Stromversorgung.
    Hierdurch wird zum Einen Energie gespart zum Anderen fällt ein Problem bei der Chipproduktion, der Spannungsversorgungsleitungen innerhalb des Chips, weg~\cite{AFM_3}.
    Auch die Frequenz der Magnonen ist größer als bei ihren Konkurrenten der ferromagnetischen Übergangsmetalloxide~\cite{AFM_5}.
    Heutzutage finden Antiferromagneten als eigene interaktive Lage bei kleineren Leseköpfen von Festplatten Anwendung, welche auf dem Riesenmagnetowiderstand (GMR, \textit{giant magnetoresistance}) beruhen.
    Sie dienen nicht mehr nur als Fixierungslage für eine ferromagnetische Schicht sondern eigenständig in Kombination mit Molekülen und einer ferromagnetischen Lage~\cite{bagrets_single_2012}.

    % Ansatz und Systeme
    Nachdem für chemisorbierte Moleküle auf Ferromagneten bereits Spin polarisierte Zustände gefunden wurden \cite{IF_16} ergeben erste theoretische Berechnungen dies für antiferromagnetische Oberflächen.
    Dort zeigen sich bei der Kombination von Molekülen und Antiferromagneten lokale Spinpolarisationen, welche die Kerneigenschaft von Spinterfaces darstellt~\cite{AFM_2}.
    Mit dem Ansatz der Kopplung von Antiferromagneten und Molekülen beschäftigt sich das europäische Forschungsprojekt Sinfonia~\cite{SINFONIA}.
    Hierbei sollen sich die Moleküle auf einer antiferromagnetischen Oberfläche regelmäßig anordnen, um diese dann gezielt zu manipulieren.
    Ferner sollen diese genutzt werden, um Spinwellen, so genannte Magnonen, zu generieren und detektieren.

    Für das Verständnis über die ablaufenden Prozesse, wie die Molekül-Substrat-Wechselwirkung an der Grenzfläche sind die Informationen über die elektronische Struktur unabdingbar.
    Um solche selbstanordnenden Moleküle hinsichtlich ihrer elektronischen Struktur zu untersuchen ist die Verwendung der Photoemissionorbitaltomographie eine starke Methode.
    % Um solche selbstanordnenden Moleküle hinsichtlich ihrer elektronischen Struktur zu untersuchen ist die Verwendung der Photoemissionorbitaltomographie, impulsaufgelösten Photoemissionsmessungen mit ab initio Berechnungen mit Hilfe der Dichtefunktionaltheorie durchführt, welche eine starke Methode.
    Sie liefert die eindeutige Identifizierung der beteiligten Molekülorbitale und deren azimutalen Ausrichtung~\cite{MM_2, MM_5}.
    
    So werden für den Einsatz in organischen Feldeffekttransistoren eine Schicht aus isolierendem Material, wie zum Beispiel \ce{NiO} oder \ce{FeO} und Molekülen, wie dem Pentacen benötigt~\cite{5A_13}.
    Das Aufbringen der Moleküle auf ein antiferromagnetisches Substrat wie dem Eisenmonooxid ist interessant, da Eisen ein häufig eingesetztes Material ist und in einer großen Anzahl zur Verfügung steht. % kann zu neuen Phänomenen und Eigenschaften führen.
    Die Erweiterung auf die Kombination mit einer polaren und somit reaktiven antiferromagnetischen Oberfläche wie die \ce{NiO} (111)-Oberfläche stellt eine neue Idee dar~\cite{cappus_hydroxyl_1993}.
    Auf dieser Oberfläche sind alle Spins in dieselbe Richtung ausgerichtet, was einen immensen Vorteil bei der Manipulation bieten könnte.
    Das untersuchte Pentacen wird heute schon in einigen organisch elektronischen Bauteilen eingesetzt~\cite{5A_4}.
    In Folge dessen werden in dieser Arbeit erste Charakterisierungen für potenzielle Substrate und Moleküle untersucht.

    % Um die Art der Interaktion zu bestimmen um die Performance zu verbessern eignet sich die Molekülorbitaltomographie besonders gut.
    % Dieses System kann also hinsichtlich dem großen Energiebedarf zur Kühlung in Rechenzentern diesen senken.
    % Aktuelle Entwicklungen der Halbleiterindustrie auf CMOS Basis sind komm noch in ihrer Größe minimierbar, auch hier könnte das System zu einer kleineren Bauweise beitragen. 
    % Somit wären gleich mehrer Probleme der IT-Industrie gleichzeitig gelöst.
    % 
    % NiO zeigt bereits Eigenschaften des Katalisators~\cite{kunz_chemisorption_1985}, eine Erweiterung und Verbesserung mit Hilfe der Moleküle wäre denkbar.
% \begin{itemize}
%     \item the ferromagnetic/organic interface of organic based spintronic devices has been target of research because evidence was found for its excessive impact on the overall device performances [15, 16, 17]. (DJ)
%     \item PEN in OFETS (\url{https://www.sciencedirect.com/science/article/abs/pii/S0065272518300357})
% \end{itemize}



