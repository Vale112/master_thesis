\chapter{Einleitung}
    % Wieso
    Die Anzahl an Menschen, die digitale Geräte wie etwa ein Smartphone nutzen, steigt stetig~\cite{Statista}.
    Damit steigt gleichermaßen der Energie- und Materialbedarf in der Technologiebranche.
    Allerdings gibt es kaum noch Ausbaupotential von Effizienz, Bauteilgröße und Minimierung der Leistungsaufnahme in der herkömmlichen Halbleitertechnik.
    Schon seit einigen Jahren werden neue Materialklassen wie Kombinationen aus organischen und anorganischen Materialien in modernen Smartphones verbaut~\cite{MAC}.
    Diese OLEDs (\textit{organic light emitting diodes}) gehören wie die OFETs (\textit{organic field effect transistors}) zu der organischen Halbleitertechnik.
    Durch die Integration dieser Komponenten in bestehende Systeme und die Überwindung von Grenzen der heutigen Halbleitertechnik bezüglich Effizenz, Größe und Gewicht, hat dieser Forschungsbereich in den letzten Jahrzehnten immer mehr Aufmerksamkeit auf sich gezogen~\cite{Uni-Tübingen}.
    Ebenfalls sorgt der immer größere Bedarf und damit steigende Preise und Engpässe bei Halbleitern für die Nachfrage nach alternativen Materialien~\cite{Idealo}.

    % Grenzfläche
    Bei solchen organisch-anorganischen Grenzflächen sind die Eigenschaften der beiden Materialien eng miteinander verknüpft.
    Die Zustände der Moleküle hybridisieren mit den Zuständen der Oberfläche, sodass die Eigenschaften dieser Grenzfläche für das Funktionsverständnis an Bedeutung gewinnen~\cite{5A_12, kroemer_nobel_2001}.
    Dabei ist die Verwendung organisch-anorganischen Materialkombinationen, aufgrund der großen Vielzahl an zur Verfügung stehenden Materialien und der damit verbundenen Anpassbarkeit viel versprechend.
    % Hier lässt sich zum Beispiel die Ionisationsenergie des Moleküls, ebenso wie die Austrittsarbei des Substrates für die Wechselwirkung mit dem Substrat und die Bandlücke für eine optische Anregung optimieren.
    Es hat sich gezeigt, dass sich die magnetischen Eigenschaften der Moleküle, wie auch der Grenzfläche durch die entsprechende Wahl der Substanzen und Schichtdicken beeinflussen lassen~\cite{IF_16}.
    Durch die große Variation von organischen Molekülen in Bezug auf ihre Struktur und Zusammensetzung lassen sich die entsprechenden elektronischen Eigenschaften gezielt steuern~\cite{scholl_chapter_2018}.
    % Auch die geringe Größe der Moleküle ermöglicht eine weiter Minimierung der elektronischen Komponenten.
    Da selbiges auch für die Übergangsmetalloxide gilt~\cite{5A_4}, ist die Erforschung der elektronischen und magnetischen Eigenschaften dieser Schichtsysteme im Fokus vieler Forschungsprojekte.


    Ein mögliches Ziel ist es, eine externe Eingabe über die Moleküle umzuwandeln und an das Substrat weiter zu geben, um diese dann nutzbar zu machen.
    Hierzu sollen mittels optischer Stimulation oder einem äußeren Feld die chemischen und physikalischen Eigenschaften des Moleküls beeinflusst werden.
    Aufgrund der Wechselwirkung an der Grenzfläche können diese Manipulationen an das Substrat weiter gegeben werden~\cite{IF_16}.
    So könnten zum Beispiel durch die gezielte Manipulation von hybridisierten Zuständen der organisch-anorganischen antiferromagnetischen Systeme Spinwellen mithilfe optischer Anregung erzeugt und ausgelesen werden~\cite{AFM_2,AFM_1}.

    % Spin und AFM -> TMOs
    Die Kopplung an die magnetischen Eigenschaften bietet den Vorteil, dass bei einem Transport der Informationen über Spins keinen Ladungstransport erforderlich ist und somit keine Joulsche-Wärme produziert wird~\cite{AFM_3}.
    Der Transport der Informationen über Spinwellen ist dadurch gleichzeitig schneller.
    Dieser Ansatz wird in der aktuellen Forschung verfolgt, um die geringe Spin-Bahn-Kopplung innerhalb der Moleküle mit den Spins im Festkörper zu verbinden~\cite{xiong_giant_2004}.
    Dabei standen bisher vermehrt ferromagnetisch-organische Grenzflächen bei der Forschung nach Spin-elektronischen Bauteilen im Fokus~\cite{ma-DJ, AFM_5}.
    Nachdem für chemisorbierte Moleküle auf Ferromagneten bereits Spin polarisierte Zustände gefunden wurden \cite{IF_16} ergeben erste theoretische Berechnungen dies ebenfalls für antiferromagnetische Oberflächen.
    Dort zeigen sich bei der Kombination von Molekülen und Antiferromagneten lokale Spinpolarisationen an der Grenzfläche, welche die Kerneigenschaft von Spinterfaces darstellen~\cite{AFM_2}.
    % Hierdurch lässt sich das elektronische Verhalten durch Manipulation der Elektronen Spins beeinflussen, welche im Bereich der Spin Elektronik zum Einsatz kommen.
    Hinzukommend können isolierende Antiferromagneten einen elektronischen Spin-Strom in einen magnonischen Spin-Strom umwandeln~\cite{AFM_1}.
    Da die Frequenzen von Spinwellen in Antiferromagneten im \si{\tera\hertz}-Bereich liegen, kann dadurch die Taktung elektronischer Geräte im Vergleich zur heutigen Zeit noch einmal deutlich erhöht werden~\cite{AFM_5}.
    Dabei zeigt die Verwendung von antiferromagnetischen Übergangsmetalloxiden eine größere Stabilität hinsichtlich ihrer Magnetisierung und schnellere Dynamiken als es für Ferromagneten bekannt ist~\cite{AFM_1}.
    
    % Der Riesenmagnetowiderstand (GMR, \textit{giant magnetoresistance}) dient in den Lesköpfen von Festplatten zur Auslesung der magnetischen Domänen.
    % Bisher wurden Antiferromagneten meist nur als Fixierungslage, die die Magnetisierungsrichtung der hart magnetisierbaren ferromagnetischen Lage gegen ein äußeres Feld festhält, eingesetzt.
    % Neuerdings können Antiferromagneten diese Lage direkt darstellen, was zu einer Reduzierung der Bauteilgröße führt.
    % Ferner wird die bisherige nicht magnetische anorganische Schicht durch eine organische Lage ersetzt~\cite{bagrets_single_2012}.

    Für Anwendungen ist die Lage der Molekülzustände hinsichtlich der Fermikante des Substrates von großer Bedeutung (Energieniveauanpassung).
    Bei der Erforschung von organisch-anorganischen Systemen stellt die Energieniveauanpassung eine wichtigen Parameter dar, da diese den Ladungstransfer beeinflusst.
    Eine besondere Rolle spielt dabei die Austrittsarbeit des verwendeten Substrates, welche sich durch gezielte Eingriffe in den Präparationsprozess der Übergangsmetalloxide anpassen lässt~\cite{5A_4, 5A_3, IF_11}.
    %  sowie die Ionisationsenergie des Moleküls welche die Bindungsenergien der Molekülorbitale an der Grenzfläche definieren~\cite{5A_3}.
    Heutzutage werden Übergangsmetalloxide bereits eingesetzt um als Zwischenschichten zwischen Elektrode und organischer Schicht in organischen Bauteilen den Kontaktwiderstand zu reduzieren.
    % Ebenfalls ermöglichen diese eine passende Energieniveauanpassung der Moleküle, da durch gezielte Eingriffe in den Präparationsprozess sich die Austrittsarbeit der Übergangsmetalloxide anpassen lässt~\cite{5A_4, IF_11}.
    % Um die Vorteile antiferromagnetischer Materialien und die Anpassbarkeit der Austrittsarbeit zu kombinieren eignen sich Übergangsmetalloxide.
    In dieser Arbeit wird die Wechselwirkung zwischen Antiferromagneten und dem Molekül Pentacen untersucht, welches aufgrund seiner hohen Elektronenbeweglichkeit heute schon in einigen organisch elektronischen Bauteilen eingesetzt wird~\cite{5A_4, 5A_13}.
    Dieses besteht aus fünf aneinander verschmolzenen Phenylringen~\cite{MM_2} und zeigt auf zahlreichen Oberflächen Selbstanordnung~\cite{5A_4, 5A_1, 5A_6, 5A_10, 5A_5, 5A_9}.
    So wurde bei diesen auch bereits Spin-polarisierte Zustände bei der Adsorption auf ferromagnetischen Inseln aus Cobald auf Kupfer gefunden~\cite{chu_spin-dependent_2015}.

    Wichtig für eine spätere Anwendung in kommerziellen Beuteilen ist dabei auch die Fähigkeit der Moleküle, sich auf bestimmten Oberflächen selbst anzuordnen.
    Zur Identifizierung vielversprechender Molekül-Substrat-Kombinationen sind zunächst die Wechselwirkungen zwischen Molekül und Substrat sowie deren elektronische und geometrische Struktur an der Grenzfläche zu verstehen.
    % Regelmäßig anordnende Moleküle auf Oberflächen stellen dabei einen Teilbereich dar, der z.B. für die Datenspeicherung von Nutzen sein könnte, da die gezielte Manipulation einzelner Moleküle ermöglicht wird.
    Um die Moleküle hinsichtlich ihrer elektronischen Struktur zu untersuchen, ist die Verwendung der Photoemissionsorbitaltomographie eine passende Methode.
    Sie ermöglicht dabei die eindeutige Identifizierung der beteiligten Molekülorbitale und kann so Aufschluss über die Energieniveauanpassung bieten~\cite{MM_2, MM_5}. % und deren Ausrichtung auf dem Substrat ermöglichen
    % Ebenfalls gibt sie dabei Aufschluss über die Energieniveauanpassung.
    % Der Einfluss der Grenzfläche auf die elektronische Struktur des Substrates kann mit der Photoelektronenemissionsmikroskopie untersucht werde, mit ihr lässt sich so die Bandstruktur der Oberfläche gewinnen.

    Nickeloxid und Eisenmonooxid sind zwei Beispiele antiferromagnetischer Übergangsmetalloxide, die in dieser Arbeit untersucht werden.
    Das Nickeloxid wurde auf eine Goldoberfläche gewachsen, da ihre Gitterkonstanten um nur \SI{5}{\percent} voneinander abweichen.
    Dabei bildet sich eine instabile, polare und reaktive (111)-Oberfläche aus, wobei die Spins in der obersten Lage eine gleichgerichtete Orientierung einnehmen~\cite{cappus_hydroxyl_1993}.
    Das Eisenmonooxid wurde auf eine passivierte Eisenoberfläche aufgebracht, da deren Gitterkonstante nur um \SI{6}{\percent} abweicht.
    Diese (100)-Oberfläche ist stabil und benachbarte Spins sind an der Oberfläche antiparallel orientiert.
    Auf beiden Substraten wurde anschließend noch eine Monolage Pentacen aufgebracht und mittels Photoemissionsorbitaltomographie charakterisiert.
    
    % So werden für den Einsatz in organischen Feldeffekttransistoren eine Schicht aus isolierendem Material, wie zum Beispiel \ce{NiO} oder \ce{FeO} und Molekülen, wie dem Pentacen benötigt~\cite{5A_13}.

    % NiO zeigt bereits Eigenschaften des Katalisators~\cite{kunz_chemisorption_1985}, eine Erweiterung und Verbesserung mit Hilfe der Moleküle wäre denkbar.
% \begin{itemize}
%     \item the ferromagnetic/organic interface of organic based spintronic devices has been target of research because evidence was found for its excessive impact on the overall device performances [15, 16, 17]. (DJ)
%     \item PEN in OFETS (\url{https://www.sciencedirect.com/science/article/abs/pii/S0065272518300357})
% \end{itemize}

% Wie schon der Nobelpreisträger H. Kroemer im Jahre 2000 sagte, ist die Grenzfläche das Bauteil (\textit{"The interface is the device"}).
% Die Verwendung von Molekülen, die aus leichten Atomen bestehen, reduziert des Weiteren das Gewicht der entsprechenden Bauteile, was besonders für mobile Geräte interessant ist.
% Durch den Verzicht auf giftige Schwermetalle und andere Substanzen sind solche organischen Materialien umweltfreundlicher als ihre Konkurrenten der heutigen Technik~\cite{scholl_chapter_2018}.
% Durch die Verwendung von Magnonen entfällt die Stromversorgung.
% Hierdurch wird zum einen Energie gespart zum anderen fällt ein Problem bei der Chipproduktion, der Spannungsversorgungsleitungen innerhalb des Chips, weg~\cite{AFM_3}.
% Heutzutage finden Antiferromagneten als eigene interaktive Lage bei kleineren Leseköpfen von Festplatten Anwendung, welche auf dem Riesenmagnetowiderstand (GMR, \textit{giant magnetoresistance}) beruhen.
% Sie dienen nicht mehr nur als Fixierungslage für eine ferromagnetische Schicht sondern eigenständig in Kombination mit Molekülen und einer ferromagnetischen Lage~\cite{bagrets_single_2012}.
% Der Energieaufwand für Anregungen in den organischen Materialien ist geringer als in den anorganischen Materialien und kann somit einfacher realisiert werden.