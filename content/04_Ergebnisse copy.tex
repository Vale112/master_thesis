\chapter{Ergebnisse}
    Innerhalb diese Kapitels geht es zunächst um die Vorbereitung der Proben.
    Weiter geht es es um den Umgang mit dem dreidimensionalen Datensatz sowie dessen Bearbeitung.
    Die verschiedenen extrahierten Darstellungen werden dann aufgenommen und analysiert.
    In dieser Arbeit untersuchten Gold und Nickeloxid Proben wurden alle in dem Versuchsaufbau aus \autoref{sec:Versuchsaufbau} vorbereitet und vermessen.
    Die Eisenoxid Proben wurden an der NanoESCA Beamline am Synchrotron Elettra in Triest präperiert und charakterisiert \footnote{Näheres zu diesem Versuchsaufbau und der Behandlung der Daten kann in \cite{ma-DJ} gefunden werden.}.

    \section{Gold}
        Zunächst geht es hier um das Substrat auf dem später der antiferromagnetische Nickeloxidfilm gewachsen wird.
        Dies dient ebenfalls zur Kalibrierung der Monolage von Pentacene, da bereits bekannt ist, dass sich diese flach auf der Oberfläche ordnet.
        \begin{figure}
            \centering
            \begin{subfigure}{0.48\textwidth}
                \centering
                \includegraphics[height=5cm]{./content/pictures/Au/2021_06_08_002_Au(111)_75eV}
                \subcaption{Gold (111) bei einer Elektronenenergie von \SI{75}{\electronvolt}.}
                \label{fig:LEED_Au}
            \end{subfigure}
            \begin{subfigure}{0.48\textwidth}
                \centering
                \includegraphics[height=5cm]{./content/pictures/Au+5A/021_Au(111)+5A(40)_16eV.png}
                \subcaption{Eine Monolage Pentacene auf dem sauberen Gold bei einer Energie von \SI{16}{\electronvolt}.}
                \label{fig:LEED_Au+5A}
            \end{subfigure}
            \caption{Die LEED-Bilder für die Kalibrierung des Pentacene.}
        \label{fig:Substrate}
        \end{figure}
        Das Substrat wurde zunächste mehrer Male durch ioneninduziertes Zerstäuben und anschließendes Aufheizen gereinigt.
        Es ergibt sich eine wohldefinierte Struktur der Oberfläche, was sich ebenfalls in dem Beugungsbild der niederenergetischen Elektronen in \autoref{fig:LEED_Au} erkennen lässt.








    \section{Vorbereitung und Präperation} \label{sec:Praep}
        \begin{figure}
            \centering
            \begin{subfigure}{0.48\textwidth}
                \centering
                \includegraphics[height=5cm]{./content/pictures/Au/2021_06_08_002_Au(111)_75eV}
                \subcaption{Gold (111) bei einer Elektronenenergie von \SI{75}{\electronvolt}.}
                \label{fig:LEED_Au}
            \end{subfigure}
            \begin{subfigure}{0.48\textwidth}
                \centering
                \includegraphics[height=5cm]{./content/pictures/pFe/2021_09_07_002_passivatedFe(100)_125eV.png}
                \subcaption{Passiviertes Eisen (100) bei einer Elektronenenergie von \SI{125}{\electronvolt}.}
                \label{fig:LEED_pFe}
            \end{subfigure}
            \caption{Die LEED-Bilder für die sauberen Substrate.}
        \label{fig:Substrate}
        \end{figure}
        Zum Präperieren der Proben werden zunächst die verschiedenen Substrate durch mehrfaches ioneninduziertes Zerstäuben und aufheizen gereinigt.
        Für die Goldprobe wurde eine Spannung von \SI{2}{\kilo\volt} und ein Strom von \SI{10}{\milli\ampere}, mit anschließendenem Aufheizen auf etwa \SI{500}{\celsius} verwendet.
        Anschließend wird die Oberflächenstruktur mittels LEED überprüft, dabei ergibt sich das Bild in \autoref{fig:Substrate}.
        Es ist sind scharfe Spots zu erkennen, ebenso wie die charakteristische kleineren Spots um die Hauptspots, die von der Fischgräten-Rekonstruktion herrühren.
        Die unterschiedlichen Intensität der einzelnen Reflexe rühr ebenfalls von der Rekonstruktion her.

        Hingegen wurde bei der Eisenprobe nur ein Spannung von \SI{1}{\kilo\volt} gewählt um die Probe zu reinigen, da es sich um einen dünnen auf Magnesiumoxid gewachsenen Film handelt.
        Allerdings wurde die Probe dann auf etwa \SI{600}{\celsius} aufgeheizt.
        Um Verunreinigungen des sehr reaktiven sauberen Eisens zu vermeiden, wird die Probe anschließend direkt passiviert.
        Dies geschieht in einer Sauerstoffatmosphäre von \SI{1.3e-7}{\milli\bar} für fünf Minuten, während die Probe bei \SI{550}{\celsius} gehalten wird.
        Die Probe wird dann nocheinmal kurz auf \SI{600}{\celsius} aufgeheizt.
        Nun wird auch dessen Oberflächenbeschaffenheit kontrolliert und entsprechendes LEED-Bild ist in \autoref{fig:LEED_pFe} zu sehen.
        Sie Spots sind scharf und zeigen auch die Geometrie der Fe-p$(1 \times 1)$O Struktur.
        Die gesamten Spots sind dabei nicht zentrosymmetrisch, da die Probe verkippt ist, wodurch auch der (0,0)-Reflex zu sehen ist.
            
        \begin{figure}
            \centering
            \begin{subfigure}{0.48\textwidth}
                \centering
                \includegraphics[height=5cm]{./content/pictures/NiO/2021_06_15_019_NiO(111)_73eV_Thicklayer}
                \subcaption{Der Nickeloxidfilm (111) bei einer Elektronenenergie von \SI{73}{\electronvolt}.}
            \end{subfigure}
            \begin{subfigure}{0.48\textwidth}
                \centering
                \includegraphics[height=5cm]{./content/pictures/FeO/2021_09_09_001_FeO_125eV.png}
                \subcaption{Eisenoxid (100) bei der Elektronenenergie von \SI{125}{\electronvolt}.}
            \end{subfigure}
            \caption{Bilder der Beugung niederenergetischer Elektronen für die gewachsenen Filme.}
        \label{fig:Filme}
        \end{figure}
        Nun wurde auf die Goldprobe bei Raumtemperatur ein Nickeloxidfilm aufgebracht, dies geschieht durch das Aufdampfen von Nickel mit einer Rate von \SI{0.3}{\angstrom\per\minute} in einer Sauerstoffatmosphäre von \SI{2e-6}{\milli\bar}.
        Für den Eisenoxidfilm wird Eisen mit einer Rate von \SI{0.6}{\ML\per\minute} und einem Sauerstoffdruck von \SI{2e-7}{\milli\bar} auf die passivierte Eisenoberfläche aufgedampft.
        Dabei wird die Probe auf eine Temperatur von \SI{230}{\celsius} gehalten und anschließend bei \SI{650}{\celsius} ausgeheizt.
        Beide Filme werden erneut mittels LEED überprüft, das Eisenoxid zusätzlich mittels Augerelektronenspektroskopie.
        Die entsprechenden LEED-Bilder sind in \autoref{fig:Filme} dargestellt.

        Bei dem Nickeloxidfilm wie auch dem Eisenoxidfilm fällt auf, dass im Gegensatz zu den Substraten die Spots eher ausgewaschen scheinen.
        Deutbar ist dies als nicht perfekt geordnete Oberflächen \cite{NiO_34}.
        Die Positionen der Punkte hat sich bei dem Nickeloxidfilm im Vergleich zum Gold nicht wesentlich verändert, ihre Gitterkonstanten sind also nahezu gleich.
        Auch die Intensitäten der Reflexe beim Nickeloxid sind nun gleich groß für alle Punkte.
        Aus der gleichen Symmetrie der Spots und der Abwesenheit zusätzlicher Spots kann eine $\text{p}(2 \times 2)$ Rekonstruktion \cite{NiO_37} und Domänenbildung der (100)-Orientierung \cite{NiO_36} ausgeschlossen werden.
        Die hier wahrscheinlichste Stabilisierung der Oberfläche ist die \ce{OH-}-Terminierung mit der $\text{p}(1 \times 1)$-Rekonstruktion \cite{NiO_35}.
        Hierbei wird das Oberflächenpotential durch Reduzierung der Oberflächenladung verkleinert und die Oberfläche wird thermodynamisch stabil.

        Hingegen sind beim Eisenoxid die Intensitäten invertiert, wobei die zuvor starken Spots nicht mehr sichtbar sind.
        Trotz der gleichen Elektronenenergie von \SI{125}{\electronvolt} sind die Spots leicht nach außen gewandert.
        Dies Widerspricht sich mit der eigentlich größeren Gitterkonstante von Eisenoxid im Bezug auf die $\text{p}(1 \times 1)\ce{O}$ Überstruktur des passivierten Eisens.

        Anschließend werden die beiden nun antiferromagnnetischen Substrate vermessen und anschließend wird eine Monolage Pentacene aufgedampft.
        Nach dem Aufdampfen wurde versucht LEED-Bilder aufzunehmen, es waren allerdings nur leichte Substratspots sichtbar.
        So lässt sich schlussfolgern, dass sich die Moleküle auf der Oberfläche nicht anordnen.

        \begin{figure}
            \centering
            \includegraphics[width=0.6\textwidth]{./content/pictures/FeO/2021_09_09_001_AES_FeO.png}
            \caption{Das Augerspektrum für den Eisenoxidfilm.}
            \label{fig:Auger_FeO}
        \end{figure}
        Auch das Augerspektrum in \autoref{fig:Auger_FeO} mit dem bereits von Carpa u.A. \cite{FeO_1} entdecken Augerelektronenspektrum für Eisenoxid zeigt gute Übereinstimmung.
        Ebenfalls deutet das Peakverhältnis von dem Sauerstoffsignal bei \SI{503}{\electronvolt} und dem Signal für Eisen bei \SI{651}{\electronvolt} mit \num{2.53} auf $\ce{FeO}$ hin \cite{FeO_1, Auger}.
        

    % \section{Datenformat und Bearbeitung}
    % \begin{itemize}
    %     \item Kreios Vorgehen
    % \end{itemize}
    %     Für die Auswertung der dreidimensionalen Datenwird die Software IGOR Pro \cite{IGOR} genutzt.
    %     Alle Messungen der winkelaufgelösten Bandstruktur wurden als dreidimensionale Datensätze aufgenommen.
    %     Da die Bilder auf Grund nicht perfekt eingestellter Linsen nicht kreisrund sind, werden die Elipsen angepasst.
    %     Hierfür wird die kleiner der beiden Achsen entlang der Achse gestreckt.
    %     Ferner muss der Polarisationsfaktor unter Beachtung der Substratgeometrie kompensiert werden.
    %     Hierfür werden die Bilder jeweils um \SI{\pm120}{} gedreht und auf das ursprüngliche Bild aufaddiert.
    %     Um die gemessene kinetische Energie in die relevante Bindungsenergie umzurechnen wird bei den integrierten Spektren aus \autoref{sec:EDC} eine Faltung aus \textbf{...} an die Fermikante durchgeführt.
    %     Zusätzlich müssen die gemessenen Bilder von ihren Pixelwerten noch in die entsprechenden Impulswerte umgerechnet werden.
    %     Dies geschieht mit Hilfe der Sekundärelektronen aus dem Spektrum der Austrittsarbeit (niedrige kinetische Energie).
    %     Ihre kinetische Energie kann durch 
    %     \begin{equation}
    %         E_\text{Kin} = \frac{\hbar^2 {k_{||}}^2}{2 m}
    %         \label{eqn:WKF}
    %     \end{equation}
    %     beschrieben werden, wobei $m$ die Elektronenmasse und $k_{||}$ ihr Impuls parallel zur Oberfläche ist.
    %     Die langsamsten Elektronen treten mit einer kinetischen Energie von \SI{0}{\electronvolt} aus der Probe aus und bilden damit den unteren Punkt der Parabel in \autoref{fig:WKF}.
    %     Durch ihren parabolischen Verlauf kann nun bei einer höher liegenden Energie ein Linienprofil genommen werden.
    %     Da sich die Elektronen wie in \autoref{eqn:WKF} verhalten können damit die Pixel in entsprechende Impulswerte umgerechnet werden.


    \section{Integrierte Spektren}
    \label{sec:EDC}
        % \begin{wrapfigure}{r}{5cm}
        \begin{figure}
            \centering
            \includegraphics[width=0.6\textwidth]{./content/pictures/NiO/NiO_Filmdicke.png}
            \caption{Die integrierten Spektren für zwei verschiedene Schichtdicken von \ce{NiO}. Als Referenz dient das integrierte Spektrum von Gold.}
            \label{fig:NiO_Filmdicke}
        \end{figure}
        \begin{figure}
            \centering
            \includegraphics[width=0.7\textwidth]{./content/pictures/FeO/Fermi_FeO.png}
            \caption{Valenzbandspektrum der des \ce{FeO} bei einer Photonenenergie von \SI{64}{\electronvolt}. Zusätzlich eingetragen ist der Fit der Fermikante.}
            \label{fig:FeO}
        \end{figure}
        In der \autoref{fig:NiO_Filmdicke} ist das Valenzbandspektrum für verschiedene Schichtdicken von Nickeloxid aufgetragen. 
        Das Signal des Substrates, in diesem Fall Gold nimmt immer weiter ab und das des Nickeloxid zu. 
        Erkennbar ist somit auch die Oberflächenemfindlichkeit, dass tiefere Lagen nicht mehr erfasst werden.
        Ebenso ist in \autoref{fig:FeO} die Elektronendichtekurve für den Valenzbandbereich des \ce{FeO} aufgetragen.
        \begin{figure}
            \centering
            \includegraphics[width=0.6\textwidth]{./content/pictures/Au+5A/EDC_Au_5A.png}
            \caption{Die integrierten Spektren für reines Gold, Gold mit einer Monolage Pentacene und deren Differenz.}
            \label{fig:Au+5A}
        \end{figure}
        \begin{figure}
            \centering
            \includegraphics[width=0.6\textwidth]{./content/pictures/NiO+5A/NiO_thick_5A.png}
            \caption{Die integrierten Spektren für einen dicken Nickeloxidfilm, mit zusaätzlich einer Monolage Pentacene und deren Differenz.}
            \label{fig:Int_NiO+5A}
        \end{figure}
        % \end{wrapfigure}
        Die Elektronendichtekurve für das Nickeloxid Substrat ist gemeinsam mit dem der zusätzlich aufgebrachten Molekülen in \autoref{fig:Int_NiO+5A} dargestellt.
        Es lassen sich bei den in den Spektren erkennbare zusätzliche Merkmale erkennen die somit den Molekülen zugeordnet werden können.
        An diesen Punkten werden einzelne Bilder mit einer erhöhten Statistik aufgenommen.

        Die Bindungsenergie wurde ermittelt in dem die Photonenenergie von \SI{21.22}{\electronvolt} angenommen wurde und die Fermikante des Goldes bei \SI{16.55}{\electronvolt} gefittet wurde.
        Damit wird die Austrittsarbeit des Analysators zu \SI{4.72}{electronvolt} bestimmt, nur dies fließt dann noch in die Gleichung \ref{eqn:Photoeffekt} ein.
        %   \begin{figure}
        %     \centering
        %     \includegraphics[width=0.7\textwidth]{./content/pictures/NiO_Filmdicke.png}
        %     \caption{Das Valenzbandspektrum von Nickeloxid im vergleich zu Messungen aus der Lieteratur.}
        %     \label{fig:NiO_Filmdicke}
        % \end{figure}
        Wird die gesamte Länge des Spektrums betrachtet, also der Energieunterschied zwischen Fermikante und Ende des Sekundärelektronen, so lässt sich die Austrittsarbeit der Probe bestimmen.
        So lässt sich erkennen, dass sich die Austrittsarbeit vom Gold zu dickeren Filmen Nickeloxid zu kleineren Werten verschiebt.
        Für Gold lässt sich die Austrittsarbeit auf \SI{5.46}{\electronvolt} ermitteln, was in der selben Ornung wie Literaturwerte liegt~\cite{Hüfner}.
        Vom dünnen Nickeloxidfilm zum dickeren Nickeloxidfilm wechselt sie von \SI{4.25}{\electronvolt} zu \SI{3.90}{\electronvolt}.
        Die Austrittsarbeit des dicken Nickeloxidfilms passt \textbf{Quelle und Wert}.
        Für das Eisenoxid ergibt sich eine Austrittsarbeit von \textbf{Wert sowie Literatur}.
        \begin{itemize}
            \item Voigt ist eine Faltung aus Lorentz, der natürlichen Linienbreite, ihre Inverse ist proportinal zur Lebenszeit des Zustands. Gauß hingegen nimmt die experimentelle Verbreiterung auf.
            \item Die Fermikante wird aus einer Faltung aus Gauß und ??? gefittet. Gauß ist ebenfalls wieder für die experimentelle Verbreiterung zuständig. Die Stufenfuktion bildet die Verbreiuterung durch die Temperatur und auch die Besetzung wieder.
            \item 
        \end{itemize}

        \begin{itemize}
            \item Sieht Features später Zuordnung
            \item Ändert sich was
            \item Verbreiterung Lebenszeit, Linienbreite Photonenquelle, Thermisch, Analysator
            \item peaks in integrierten Spektren zugeordnen
        \end{itemize}

    \section{Bandstruktur}
        \begin{figure}
            \centering
            \includegraphics[width=0.7\textwidth]{./content/pictures/Au/Bandstructure_Au111.png}
            \caption{Die gemessene Bandstruktur von Gold (111).}
            \label{fig:Bandstructure_Au}
        \end{figure}
        \begin{figure}
            \centering
            \includegraphics[width=0.7\textwidth]{./content/pictures/Au/BZ_Au.png}
            \caption{Die gemessene Winkelverteilung von Gold (111) an der Fermifläche.
            Eingezeichnet in rot ist die erste Brillouinzone und drei Hochsymmetriepunkte, entlang deren Richtung der Stack für die Bandstruktur geschnitten wird.}
            \label{fig:BZ}
        \end{figure}
        In \autoref{fig:Bandstructure_Au} lässt sich die Bandstruktur erkennen, welche aus dem Stack entlang einiger Hochsymmetrierichtungen extrahiert wurde.
        Der Schnitt ist in \autoref{fig:BZ} veranschaulicht.


        \begin{itemize}
            \item Bandstruktur von Gold
            \item Bandstruktur NiO? Spin?
            \item Bandstruktur mit Molekülen - Oberflächenzustände? Extra Features
        \end{itemize}

    \section{Molekülorbitaltomographie}
        In den Bildern von Pentacene auf Nickel- und Eisenoxid lassen sich leider keine Molekülorbitale zuordnen.
        Die Ursache ist, dass wie bereits im \autoref{sec:Praep} anhand der LEED Bilder festgestellt wurde, die Moleküle sich nicht regelmäßig auf der Oberfläche anordnen.
        Ferner überlappen dann die einzelnen Merkmale im Impulsraum, sodass ein ausgewaschenes Bild entsteht.

        Wenn hingegen die Moleküle auf dem reinen Goldkristall aufgebracht werden er gibt sich eine periodische Struktur.
        Dies lässt sich anhand von LEED Bildern erkennen, ebenso in der Möglichkeit die Bilder im Impulsraum den theoretisch ermittelten Molekülorbitalen zuzuordnen, wie es in der \autoref{fig:MOT} geschehen ist.
        Hierzu wurden auf die berechneten Bilder die selben symmetrisierungs Schritte angewendet wie auf die gemessenen.

        Der größte Unterschied zwischen dem Gold und den Oxiden ist deren isolierenden Eigenschaften, es lässt sich also vermuten, dass für die Ornung Leitfähigkeit voraus gesetzt wird.
        Die theoretisch berechneten Maps wurden mit Hilfe des Python Programms \textit{kmap.py} erstellt~\cite{brandstetter_kmappy_2021}.

        \begin{itemize}
            \item Maps selbst die sich zuordnen lassen
            \item Aus den LP bestimmte zuordnung möglich?
        \end{itemize}

        \subsection{5A auf Au}
            \begin{figure}
                \centering
                \begin{subfigure}[t]{0.48\textwidth}
                    \centering
                    \includegraphics[height=5cm]{./content/pictures/Au+5A/IMAGE_2021_06_17_005_BE0_8}
                    \subcaption{Gemmesen, symmetrisiertes Bild bei einer Bindungsenergie von \SI{0.8}{\electronvolt}.}
                \end{subfigure}
                \begin{subfigure}[t]{0.48\textwidth}
                    \centering
                    \includegraphics[height=5cm]{./content/pictures/Au+5A/HOMO1_all_CT}
                    \subcaption{Theorie Oribtale mit symmetrisierung 2mal um 120 Grad gedreht und zum Ursprungsbild addiert.}
                \end{subfigure}
                \caption{Zuordnung eines Bildes zu einem der Molekülorbitale.}
                \label{fig:MOT}
            \end{figure}

        \subsection{5A auf NiO}
            \begin{figure}
                \centering
                \includegraphics[width=0.65\textwidth]{./content/pictures/NiO+5A/NiO_thick_5A_KE12_7.png}
                \caption{Map für Pentacene auf dicken Nickeloxidfilm bei einer kinetischen Energie von \SI{7.5}{\electronvolt}. Also Bindungsenergie von \SI{3.85}{\electronvolt}.}
                \label{fig:NiO+5A}
            \end{figure}
            Die Abwesenheit sehr ausgeprägter Merkmale in den impulsaufgelösten Bildern bestätigt die Annahme aus \autoref{sec:Praep}, dass sich die Moleküle auf der Oberfläche nicht regelmäßig anordnen.
            Auch wenn sich in den integrierten Spektren in \autoref{fig:NiO_Filmdicke} klar zeigt, dass sich bei einigen Energien die Intensität erhöht ist in den entsprechenden Bildern keine klare Zuordnung möglich.
            Ein Beispiel ist in \autoref{fig:NiO+5A} zu sehen, dieses Bild wurde gewählt da ein deutliches Signal im integrierten Spektrum zu erkennen ist und die entsprechende Energie nicht zu weit weg von der Fermikante liegt.

        \subsection{FeO}
            \begin{figure}
                \centering
                \includegraphics[width=0.65\textwidth]{./content/pictures/FeO/XMCD_FeO.png}
                \caption{XAS Spektren für links- und rechtzirkular polarisiertes Licht und ihre Differenz. XMCD}
                \label{fig:XMCD}
            \end{figure}
            \begin{figure}
                \centering
                \includegraphics[width=0.65\textwidth]{./content/pictures/FeO/XMLD_FeO.png}
                \caption{XAS Spektren für s- und p- polarisiertes Licht, sowie dessen Differenz. XMLD}
                \label{fig:XMLD}
            \end{figure}
            \begin{figure}
                \centering
                \includegraphics[width=0.6\textwidth]{./content/pictures/Fe/Fe3p_Fe.png}
                \caption{XPS Spektrum des $\ce{Fe}_{3\text{p}}$ Übergang des reinen Eisens. Hier passt der Fit mit einem Voigt (+Tougaard) sehr gut!}
                \label{fig:XPSFe3p_Fe}
            \end{figure}
            \begin{figure}
                \centering
                \includegraphics[width=0.6\textwidth]{./content/pictures/FeO/Fe3p_FeO.png}
                \caption{XPS Spektrum des $\ce{Fe}_{3\text{p}}$ Übergang des \ce{FeO}. Zuornung sehr schwierig, ein großer Peak bei kleinen BE und ein kleiner bei großen BE passen. Aber auch zwei etwa gleichgroße Peaks. Je nach dem wie die Parameter gewählt und festgestzt werden.}
                \label{fig:XPSFe3p_FeO}
            \end{figure}
            \begin{figure}
                \centering
                \includegraphics[width=0.6\textwidth]{./content/pictures/Fe3O4/Fe3p_Fe3O4.png}
                \caption{XPS Spektrum des $\ce{Fe}_{3\text{p}}$ Übergang des \ce{Fe3O4}. Gleiches Problem wie beim \ce{FeO} (s. \autoref{fig:XPSFe3p_Fe}) funktioniert abenfalls besser mit Tougaard als Shirley.}
                \label{fig:XPSFe3p_Fe3O4}
            \end{figure}
            \begin{figure}
                \centering
                \includegraphics[width=0.6\textwidth]{./content/pictures/FeO/O1s_FeO.png}
                \caption{XPS Spektrum des $\ce{O}_{1\text{s}}$ Übergang von \ce{FeO}. Fit mit einem Peak (kleiner Shirley) sehr gut.}
                \label{fig:XPSO1s_FeO}
            \end{figure}
            \begin{figure}
                \centering
                \includegraphics[width=0.6\textwidth]{./content/pictures/Fe3O4/O1s_Fe3O4.png}
                \caption{XPS Spektrum des $\ce{O}_{1\text{s}}$ Übergang von \ce{Fe3O4}. Fit mit zwei Peaks (kleiner Shirley) sehr gut (einer O, der andere OH - Kontamination evtl vom Aufdampfen?).}
                \label{fig:XPSO1s_Fe3O4}
            \end{figure}
            \begin{figure}
                \centering
                \includegraphics[width=0.7\textwidth]{./content/pictures/pFe/Photonenenergie.png}
                \caption{Vergleich des VB-Spektrums bei \SI{64}{\electronvolt} und \SI{40}{\electronvolt} Photonenenergie des passivierten Eisens.}
                \label{fig:Photonenenergie}
            \end{figure}

            \begin{itemize}
                \item Zuordnung des Fe3p Peaks schwer. Manche sagen sechs Peaks voraus \cite{FeO_14, FeO_17, FeO_15} wegen der Finestruktur, andere zwei (Fe2+, Fe3+) \cite{FeO_15, FeO_11, FeO_10, FeO_7} manche drei (Fe2+, Fe3+tetra, Fe3+octa) \cite{FeO_12}. es kommt dabei auf das Material an, zum Identifizieren ist es also schwierig durch den einen Peak.
                \item Die Bindungsnergie des $\ce{Fe}^{2+}$ liegt bei \SI{53.7}{\electronvolt} und die des $\ce{Fe}^{3+}$ bei \SI{55.6}{\electronvolt} \cite{FeO_7}.
                \item Der Peak des reinen Eisens (\autoref{fig:XPSFe3p_FeO}) ist nicht dem Fe2+ oder Fe3+ zuzuornden. FeO besitzt hingegen nur Fe2+, ist aber recht anfällig weiter zu oxidieren, so dass sich ein Fe3O4 (Fe2+, Fe3+tetra, Fe3+octa) oder Fe2O3 (nur Fe3+) Film bilden kann. 
                \item Bei dem O1s Peak ist man sich einig, dass sich der Hauptpeak nicht verschiebt, eine Unterscheidung der verschiednen Eisenoxide ist also damit nicht möhlich. es können zusätzlich Peaks durch Adsorbate (z.B. OH) auftauchen.
                \item Der  $\ce{O}_{1\text{s}}$ O2- Peak liegt bei \SI{529.7}{\electronvolt} \cite{FeO_9} OH- bei \SI{531.2}{\electronvolt} \cite{FeO_9}., \SI{530.1}{\electronvolt} \cite{FeO_15} (unabhängig welches Eisenoxid).
                \item XAS-Messungen mit Teilelektronen ausbeute von Sekundärelektronen bei einer Energie von $E_\text{Kin} = \SI{7}{\electronvolt}$.
                \item XAS Messungen wurden duch Multiplikation an Preedge ausgerichtet, dann wurde ein linear Untergrund abgezogen (pre-edge gefittet) und anschließend die Preedge auf Null gesetzt und die Postedge auf 1 (durch Division)
                \item Das XMCD Signal sollte für L3 und L2 umegkehrt sein, wenn es Ferromagnetisch ist, Signal lässt sich aber nur bei L3 erkennen.
                \item Auch XMLD für Antiferromagnetismus auf Grund der nicht spärischen Oribtale durch die Spin-Bahn-Kopplung (Spins ausgerichtet) \cite{stohr_magnetism_2006} - kann auch eben sein, dass die Spins nicht ausgerichtet waren. T war unter Neel Temperatur.
                \item Fermikante beim Isolator wie \ce{FeO} (was vermutet wird vorliegen zu haben) ist schwierig (Austrittsarbeit des Analyseres liegt bei \SI{5.07}{\electronvolt}). Also bei reinem Eisen gefittet (ebenfalls schwierig da mit Peak des VB überlappt) und damit über die Austrittsarbeit des Analysators \SI{4.5}{\electronvolt} und der Photonenenergie die Bindungsenergie bestimmt.
                \item FeO kann durch ioneninduzierte Zerstäubung aus anderen Eisenoxiden gewonnen werden, da der Wirkungsquerschnitt für Sauerstoff dabei größer ist und somit diese in der Konzentration reduziert werden. \cite{FeO_12}
                      Dann verschwinden auch Signale des Fe3+ \cite{FeO_15}, Einen Einfluss auf das FeO hat das Sputtern wohl nicht \cite{FeO_12, FeO_15}.
                \item Abschnitt des Spektrums des FeO bei \SI{-1.02}{\electronvolt}, Beginn bei \SI{58.93}{\electronvolt} -> Austrittsarbeit des \ce{FeO} \SI{4.05}{\electronvolt}, da $h\nu = \SI{64}{\electronvolt}$
            \end{itemize}
            % Der $\ce{Fe}_{3\text{p}} (\SI{54.52}{\electronvolt})$ wie auch der $\ce{O}_{1\text{s}} (\SI{529.15}{\electronvolt})$ lassen sich durch einen Peak fitten.

            \begin{figure}
                \centering
                \includegraphics[width=0.7\textwidth]{./content/pictures/Fe/BZ_Fe.png}
                \caption{Die Brillouinzone des reinen Eisens bei einer Bindungsenergie von \SI{0.3}{\electronvolt}.
                Eigezeichnet sind auch die Vektoren, sowie einige Hochsymmetrierichtungen.
                Die Kantenlänge der BZ ist $\frac{2\pi}{a} = \SI[per-mode=reciprocal]{2.20}{\per\angstrom}$.
                Im Zentrum liegt der $\overline{\Gamma}$-Punkt, der Abstand zum $\overline{X}$-Punkt ist $\frac{2\pi}{2 \cdot a} = \SI[per-mode=reciprocal]{1.10}{\per\angstrom}$ und zum $\overline{M}$-Punkt $\frac{2\pi}{2 \cdot a} \sqrt{2} = \SI[per-mode=reciprocal]{1.55}{\per\angstrom}$.}
                \label{fig:BZ_Fe}
            \end{figure}
            \begin{figure}
                \centering
                \includegraphics[width=0.7\textwidth]{./content/pictures/FeO/BZ_FeO.png}
                \caption{Die Brillouinzone des Eisensoxid bei einer Bindungsenergie von \SI{1.95}{\electronvolt}.
                Eigezeichnet sind auch die Vektoren, sowie einige Hochsymmetrierichtungen.
                Die Kantenlänge der BZ ist $\frac{2\pi}{a}\sqrt{2} = \SI[per-mode=reciprocal]{2.89}{\per\angstrom}$.
                Im Zentrum liegt der $\overline{\Gamma}$-Punkt, der Abstand zum $\overline{X}$-Punkt ist $\frac{2\pi}{2 \cdot a}\sqrt{2} = \SI[per-mode=reciprocal]{1.45}{\per\angstrom}$ und zum $\overline{M}$-Punkt $\frac{2\pi}{a} = \SI[per-mode=reciprocal]{2.05}{\per\angstrom}$ \cite{Hüfner}.
                \textbf{Problem with px-> k, WKF was Driffting up and down}}
                \label{fig:BZ_FeO}
            \end{figure}

            \subsection{5A auf FeO}
                Im Widerspruch zu den Interpretation aus \autoref{sec:Praep} wo das nicht Vorhandensein die Interpretation zulies, dass sich die Moleküle nicht auf der Oberfläche ordnen sind in den Tomographiebildern merkmale von Molekülen zu erkennen.
                Diese Merkmale können sich nur ausbilden, wenn sich die Moleküle regelmäßig und in gleicher Orientierung anordnen.
                Anderenfalls würden sich die Photoemissionströme überlagern und es gäbe verschwommene Bilder.

                \begin{itemize}
                    \item Schaut man sich die integrierten Spektren für zwei verschiedene Photonenenergien an (s. \autoref{fig:Photonenenergie}), so erkennt man schon mein passivierten Eisenoxid deutliche Unterschiede.
                    \item Auch die Stacks zeigen markante Unterschiede und zeigen bei den gleichen Energien wie die HOMO bis HOMO-2 Orbitale sehr ähnliche Signale wie mit den Molekülen. Eindeutige Zurnung ohne Refernzmessung nicht möglich, die für die Photonenenergie von \SI{40}{\electronvolt} mit der auch die Moleküle vermessen wurden fehlt.
                \end{itemize}
                 
                \begin{figure}
                    \centering
                    \begin{subfigure}[t]{0.48\textwidth}
                        \centering
                        \includegraphics[height=5cm]{./content/pictures/FeO+5A/FeO_5A_34_80eV.png}
                        \subcaption{Das Bild für eine kinetische Energie von \SI{34.80}{\electronvolt}, also \SI{0.70}{\electronvolt} Bindungsenergie.}
                    \end{subfigure}
                    \begin{subfigure}[t]{0.48\textwidth}
                        \centering
                        \includegraphics[height=5cm]{./content/pictures/FeO+5A/MO_LUMO_RT_RT.png}
                        \subcaption{Das LUMO mit Symmetrisierung zweier um \SI{90}{\degree} verdrehten Übergitter.}
                    \end{subfigure}
                    \caption{Vergleich der gemessenen Intensitätsverteilung mit der des symmetrisierten LUMO.}
                    \label{fig:FeO5A1}
                \end{figure}
                \begin{figure}
                    \centering
                    \begin{subfigure}[t]{0.48\textwidth}
                        \centering
                        \includegraphics[height=5cm]{./content/pictures/FeO+5A/FeO_5A_33_75eV.png}
                        \subcaption{Das Bild für eine kinetische Energie von \SI{33.75}{\electronvolt}, also \SI{1.75}{\electronvolt} Bindungsenergie.}
                    \end{subfigure}
                    \begin{subfigure}[t]{0.48\textwidth}
                        \centering
                        \includegraphics[height=5cm]{./content/pictures/FeO+5A/MO_HOMO_RT_RT.png}
                        \subcaption{Das HOMO mit Symmetrisierung zweier um \SI{90}{\degree} verdrehten Übergitter.}
                    \end{subfigure}
                    \caption{Vergleich der gemessenen Intensitätsverteilung mit der des symmetrisierten HOMO.}
                    \label{fig:FeO5A2}
                \end{figure}
                \begin{figure}
                    \centering
                    \begin{subfigure}[t]{0.48\textwidth}
                        \centering
                        \includegraphics[height=5cm]{./content/pictures/FeO+5A/FeO_5A_32_15eV.png}
                        \subcaption{Das Bild für eine kinetische Energie von \SI{32.15}{\electronvolt}, also \SI{3.35}{\electronvolt} Bindungsenergie.}
                    \end{subfigure}
                    \begin{subfigure}[t]{0.48\textwidth}
                        \centering
                        \includegraphics[height=5cm]{./content/pictures/FeO+5A/MO_HOMO1_RT_RT.png}
                        \subcaption{Das HOMO-1 mit Symmetrisierung zweier um \SI{90}{\degree} verdrehten Übergitter.}
                    \end{subfigure}
                    \caption{Vergleich der gemessenen Intensitätsverteilung mit der des symmetrisierten HOMO-1.}
                    \label{fig:FeO5A3}
                \end{figure}
                \begin{figure}
                    \centering
                    \begin{subfigure}[t]{0.48\textwidth}
                        \centering
                        \includegraphics[height=5cm]{./content/pictures/FeO+5A/FeO_5A_30_95eV.png}
                        \subcaption{Das Bild für eine kinetische Energie von \SI{30.95}{\electronvolt}, also \SI{4.55}{\electronvolt} Bindungsenergie.}
                    \end{subfigure}
                    \begin{subfigure}[t]{0.48\textwidth}
                        \centering
                        \includegraphics[height=5cm]{./content/pictures/FeO+5A/MO_HOMO2_RT_RT.png}
                        \subcaption{Das HOMO-2 mit Symmetrisierung zweier um \SI{90}{\degree} verdrehten Übergitter.}
                    \end{subfigure}
                    \caption{Vergleich der gemessenen Intensitätsverteilung mit der des symmetrisierten HOMO-2.}
                    \label{fig:FeO5A4}
                \end{figure}

                \begin{itemize}
                    \item Wie liegen die Moleküle zur Substart Achse
                    \item Chemisorption wegen dem besetzten LUMO
                \end{itemize}

