\chapter{Ergebnisse}
    Innerhalb diese Kapitels geht es zunöchst um die Vorbereitung der Proben.
    Weiter geht es es um den Umgang mit dem dreidimensionalen Datensatz sowie dessen Bearbeitung.
    Die verschiedenen extrahierten Darstellungen werden dann aufgenommen und analysiert.
    In dieser Arbeit untersuchten Proben wurden alle in dem Versuchsaufbau aus Kapitel \ref{sec:Versuchsaufbau} vorbereitet und vermessen.

    \section{Vorbereitung und Präperation}
        \begin{itemize}
            \item Sputtern und annealen
            \item Gold
            \item NiO
            \item Moleküle
            \item LEED-Bilder
        \end{itemize}

    \section{Datenformat und Bearbeitung}
        \begin{itemize}
            \item 3D Cubes
            \item BE 
            \item px to k
            \item symmetrisieren
        \end{itemize}

    \section{Integrierte Spektren}
        \begin{itemize}
            \item Sieht Features später Zuordnung
            \item Ändert sich was
        \end{itemize}

    \section{Bandstruktur}
        \begin{itemize}
            \item Bandstruktur von Gold
            \item Bandstruktur NiO? Spin?
            \item Bandstruktur mit Molekülen - Oberflächenzustände? Extra Features
        \end{itemize}

    \section{Maps}
        \begin{itemize}
            \item Maps selbst die sich zuordnen lassen
            \item Aus den LP bestimmte zuordnung möglich?
        \end{itemize}