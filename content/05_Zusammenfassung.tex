\chapter{Zusammenfassung und Ausblick}
    Zusammenfassend lässt sich sagen, dass sich die Präperation von wohldefinierten dünnen Übergangsmetalloxiden als recht schwierig erweist und genauere Untersuchungen benötigt werden.
    Besonders hinsichtlich von Kontaminationen und Stochimetrie, sowie der Oberflächenbeschaffenheit sind Verbesserungen von nöten.
    Bevor weitere Untersuchung der Wechselwirkung mit Molekülen durchgeührt werden, sollten zunächste die Substrate vollständig charakterisiert und reproduzierbar präperiert werden.
    Denn wie bei ferromagnetisch Systemen sind die Eigenschaften des zugrundeliegenden Systems entscheidend für den Spintransport~\cite{IF_16}.

    Nickeloxidfilme lassen sich ohne große Probleme durch aufdampfen in einem Sauerstoffdruck erzielen.
    Hierbei wird die polare und instabile (111)-Oberfläche durch eine \ce{OH-}-Kontamination stabilisiert.
    Beim Aufdampfen von Eisen in einer Sauerstoffatmosphäre kommt es zunächst zu einer gemischten Eisenoxidphase.
    Durch dan Einfluss des ioneninduzierten Zerstäuben führt dies zur Phase des \ce{FeO}.
    
    Für das Nickeloxis zeigt sich für unterschiedliche Schichtdicken verschiedenen Energienieveauanpassungen der Molekülorbitale.
    So wurde für das Nickeloxidfilm ebenfalls gezeigt, dass eine zu reaktive Oberfläche die Selbstanordnen behindern kann.
    Wohin die weniger reaktive Oberfläche des Wüstit die Selbstanordnen unterstütz, was sich mittels der Molekülorbitaltomographie bestätigen lässt.
    Ferner wird trotz der isolierenden Eigenschaft einen Ladungsaustausch mit den Molekülen begünstigt und das LUMO wird besetzt.

    weitergehende Untersuchungen im Bereich von Spin aufgelöste Messungen wären denkbar um den Effekt des Antiferromagneten und den Einfluss auf die Molekülbesetzung zu interpretieren.
    So eignet sich die Spin aufgelöste Molekülorbitaltomographie zur charakterisieren der Molekülorbitale.
    Mit der Messung von Röntgenabsorptionsspektren für verschiedene Polarisationen kann die Ausrichtung der Moleküle und die magnetischen Eigenschaften der einzelnen Beteiligten bestimmt werden.
    Unterschiedliche Schichtdicken und Austrittsarbeiten durch die Sauerstoffkonzentration sollten Beachtung finden um die Energieniveauanpassung weiter gehend zu otimiereen~\cite{IF_8}.
    Auch die Absorptionsstärke lässt sich hierdurch, sowie der Wahl der Oberflächenorientierung beeinflussen.
