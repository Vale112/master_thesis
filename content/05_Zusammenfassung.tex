\chapter{Zusammenfassung und Ausblick}
    Durch das Sauerstoff begleitete Aufdampfen von Nickel lässt sich Nickeloxid aufbringen.
    Hierbei wird die polare und instabile (111)-Oberfläche durch eine \ce{OH-}-Terminierung stabilisiert.
    Beim Aufdampfen von Eisen in einer Sauerstoffatmosphäre kommt es zunächst zu einer gemischten Eisenoxidphase.
    Durch den Einfluss des ioneninduzierten Zerstäubens führt dies zur Phase des $\ce{Fe}_\text{x}\ce{O}$.
    
    Für das Nickeloxid zeigt sich für unterschiedliche Schichtdicken verschiedenen Energienieveauanpassungen der Molekülorbitale.
    So wurde für den Nickeloxidfilm ebenfalls gezeigt, dass eine instabile und reaktive Oberfläche die Selbstanordnung von Molekülen behindern kann.
    Wohin die weniger reaktive und geordnete Oberfläche des Wüstit die Selbstanordnung unterstütz, was sich mittels der Photoemissionsorbitaltomographie bestätigen lässt.
    Ferner wird trotz der isolierenden Eigenschaft ein Ladungsaustausch mit den Molekülen begünstigt und das LUMO wird besetzt.

    Zusammenfassend lässt sich sagen, dass sich die Präparation von wohldefinierten dünnen Übergangsmetalloxiden als recht schwierig erweist und genauere Untersuchungen benötigt werden.
    Besonders hinsichtlich von Kontaminationen und Stöchiometrie, sowie der Oberflächenbeschaffenheit sind Verbesserungen von Nöten.
    So soll zum Beispiel ein schnelles Herabkühlen die Stöchiometrie des Wüstits erhöhen und damit die Faktorisierung in \ce{Fe} und \ce{Fe3O4}-Fehlstellen unterbinden~\cite{parkinson_iron_2016}.
    Das Aufbringen der Molekühle und der Untersuchung von Wechselwirkungen mit den Substraten sollten durchgeführt werden, denn die Substarte vollständig charakterisiert sind.

    % sollten zunächste die Substrate vollständig charakterisiert und reproduzierbar präpariert werden können.
    % Das genaue Verständnis über das vorhandene System ist unabdinglich um es für die Anwendung nutzbar zu machen.
    % Denn wie bei ferromagnetisch Systemen sind die Eigenschaften des zugrundeliegenden Systems entscheidend für den Spintransport~\cite{IF_16}.

    Weitergehende Untersuchungen im Bereich von Spin aufgelösten Messungen wären denkbar, um den Effekt des Antiferromagneten und den Einfluss auf die Molekülbesetzung zu interpretieren.
    So eignet sich die Spin aufgelöste Photoemissionsorbitaltomographie zur Charakterisierung der Molekülorbitale.
    Mit der Messung von Röntgenabsorptionsspektren für verschiedene Polarisationen kann die Ausrichtung der Moleküle und die magnetischen Eigenschaften des Substartes sowie der Moleküle bestimmt werden.
    Unterschiedliche Schichtdicken und Austrittsarbeiten durch die Sauerstoffkonzentration sollten Beachtung finden, um die Energieniveauanpassung weitergehend zu optimieren~\cite{IF_8}.
    Auch die Absorptionsstärke lässt sich durch die Austrittsarbeit, sowie der Wahl der Oberflächenorientierung beeinflussen.