\chapter{Zusammenfassung und Ausblick}
    Durch das Sauerstoff begleitete Aufdampfen von Nickel lässt sich Nickeloxid aufbringen.
    Hierbei wird die polare und instabile (111)-Oberfläche durch eine \ce{OH-}-Terminierung stabilisiert.
    Beim Aufdampfen von Eisen in einer Sauerstoffatmosphäre kommt es zunächst zu einer gemischten Eisenoxidphase.
    Durch den Einfluss des ioneninduzierten Zerstäubens führt dies zur Phase des $\ce{Fe}_\text{x}\ce{O}$.
    
    Für das Nickeloxid zeigt sich für unterschiedliche Schichtdicken verschiedenen Energienieveauanpassungen der Molekülorbitale.
    So wurde für den Nickeloxidfilm ebenfalls gezeigt, dass eine instabile und reaktive Oberfläche die Selbstanordnung von Molekülen behindern kann.
    Wohingegen die weniger reaktive und geordnete Oberfläche des Wüstit die Selbstanordnung unterstützt, was sich mittels der Photoemissionsorbitaltomographie bestätigen lässt.
    Ferner wird trotz der isolierenden Eigenschaft ein Ladungsaustausch mit den Molekülen begünstigt und das LUMO wird besetzt.

    Zusammenfassend lässt sich sagen, dass sich die Präparation von wohldefinierten dünnen Übergangsmetalloxiden komplex gestaltet und genauere Untersuchungen erforderlich sind.
    Besonders hinsichtlich von Kontaminationen und Stöchiometrie sowie der Oberflächenbeschaffenheit sind Verbesserungen für zukünftige Untersuchungen notwendig.
    Ein schnelles Herabkühlen soll die Faktorisierung des Wüstits in \ce{Fe} und \ce{Fe3O4} unterbinden~\cite{parkinson_iron_2016}.
    Die Untersuchung von Wechselwirkungen von aufgedampften Molekülen auf Oberflächen sollte grundsätzlich durchgeführt werden, sobald die Substrate selbst vollständig charakterisiert sind.
    % sollten zunächste die Substrate vollständig charakterisiert und reproduzierbar präpariert werden können.
    % Das genaue Verständnis über das vorhandene System ist unabdinglich um es für die Anwendung nutzbar zu machen.
    % Denn wie bei ferromagnetisch Systemen sind die Eigenschaften des zugrundeliegenden Systems entscheidend für den Spintransport~\cite{IF_16}.

    Weitergehende Untersuchungen im Bereich von Spin aufgelösten Messungen wären denkbar, um den Effekt des Antiferromagneten und den Einfluss auf die Molekülbesetzung zu interpretieren.
    So eignet sich die Spin aufgelöste Photoemissionsorbitaltomographie zur Charakterisierung der Molekülorbitale.
    Mit der Messung von Röntgenabsorptionsspektren für verschiedene Polarisationen kann die Ausrichtung der Moleküle und die magnetischen Eigenschaften des Substartes sowie der Moleküle bestimmt werden.
    Für eine weitere Optimierung der Energieniveauanpassung sollten unterschiedliche Schichtdicken und Austrittsarbeiten beachtet werden~\cite{IF_8}.
    Auch die Absorptionsstärke lässt sich durch die Austrittsarbeit sowie durch die Wahl der Oberflächenorientierung beeinflussen.