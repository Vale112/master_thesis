\section{NiO}
In NiO successive (111) lattice planes have opposite spin alignment. Thus ”uncompensated”
surfaces can be generated at the surface of a single crystal. Within the surface the magnetic
moment has three equivalent possibilities to orient, namely in [211],[121] or [112] directions.
Together with the four equivalent (111) surfaces this results in 12 possible spin orientations, and,
thus in 12 different possibilities to form antiferromagnetic domains. To proof that the anisotropy
is due to the antiferromagnetic alignment and not due to another anisotropic effect one can
heat the sample to its Ne ́el temperature (520 K for NiO) where the magnetic contribution will
disappear. From \url{https://www.fz-juelich.de/SharedDocs/Downloads/PGI/PGI-6/EN/E9_XAS_Bechthold07_a.pdf?__blob=publicationFile}

\section{}