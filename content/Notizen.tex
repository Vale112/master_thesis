\section{NiO}
    In NiO successive (111) lattice planes have opposite spin alignment. Thus ”uncompensated”
    surfaces can be generated at the surface of a single crystal. Within the surface the magnetic
    moment has three equivalent possibilities to orient, namely in [211],[121] or [112] directions.
    Together with the four equivalent (111) surfaces this results in 12 possible spin orientations, and,
    thus in 12 different possibilities to form antiferromagnetic domains. To proof that the anisotropy
    is due to the antiferromagnetic alignment and not due to another anisotropic effect one can
    heat the sample to its Ne ́el temperature (520 K for NiO) where the magnetic contribution will
    disappear. From \url{https://www.fz-juelich.de/SharedDocs/Downloads/PGI/PGI-6/EN/E9_XAS_Bechthold07_a.pdf?__blob=publicationFile}

\section{VB der TMOs}
    Die Bandstruktur des Kristalls wird im niederenergetischen Valenzbandbereich durch die überlappenden 2p-Orbitale des Sauerstoffs dominiert.
    Zu höheren Bindungsenergien hin rührt die Bandstruktur von den überlappenden d-Orbitalen der Übergangsmetalle her.
    Dabei sind die 2p-Zustände stark besetzt wohingegen die d-Zustände nur schwach besetzt werden.
    Bei der Betrachtung der Bandstruktur der Übergangsmetalloxide ist zu beachten, dass diese eine Bandlücke zwischen Valenz- und Leitungsband aufweisen.
    Diese Bandlücke liegt dabei zwischen den besetzen Sauerstoffzuständen und den unbesetzten d-Zuständen des Übergangsmetall.
    Folglich handelt es sich also um Halbleiter oder Isolatoren.