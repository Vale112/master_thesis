\begin{table}
    \centering
    \caption{Zusammenfassung der wichtigsten Systemparameter der verwendeten Substrate.}
    \label{tab:Systeme}
    \begin{tabular}{l c c}
        \toprule
        {Materialeigenschaften} & {NiO} & {FeO} \\
        \midrule
        Kristallstruktur \cite{sebbari_uranyl_2012} & \ce{NaCl} & \ce{NaCl} \\
        Gitterkonstante & \SI{4.17}{\angstrom} \cite{sebbari_uranyl_2012} & \SI{4.308}{\angstrom} \cite{springer_database}\\
        Oberflächenorientierung & (111) & (100) \\
        % \multirow{2}{*}{Oberflächenrekonstruktion \footnote{bzgl. des Substartes}} & Domänen (100) & $\text{p}(1 \times 1)$ inv. Intensitäten\\
        % & & \\
        magn. Eigenschaft \cite{FeO_6}& antiferromagnetisch & antiferromagnetisch \\
        Neèl-Temperatur \cite{FeO_6} & \SI{525}{\kelvin} & \SI{198}{\kelvin} \\
        Austrittsarbeit & \SIrange{4.5}{5.2}{\electronvolt} \cite{poulain_electronic_2020} & \SI{3.5}{\electronvolt} \cite{FeO_28}\\
        Bandlücke & \SI{3.6}{\electronvolt} \cite{kunz_chemisorption_1985} & \SI{2.4}{\electronvolt} \cite{FeO_21}\\
        \bottomrule
    \end{tabular}
    
\end{table}
